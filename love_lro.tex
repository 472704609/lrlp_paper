%%%%%%%%%%%%%%%%%%%%%%%%%%%%%%%%%%%%%%%%%%%%%%%%%%%%%%%%%%%%%%%%%%%%%%%%%%%%
%% Author template for Operations Reseacrh (opre) for articles with no e-companion (EC)
%% Mirko Janc, Ph.D., INFORMS, mirko.janc@informs.org
%% ver. 0.95, December 2010
%%%%%%%%%%%%%%%%%%%%%%%%%%%%%%%%%%%%%%%%%%%%%%%%%%%%%%%%%%%%%%%%%%%%%%%%%%%%
%\documentclass[opre,blindrev]{informs3}
\documentclass[opre,nonblindrev]{informs3} % current default for manuscript submission

\OneAndAHalfSpacedXI % current default line spacing
%%\OneAndAHalfSpacedXII
%%\DoubleSpacedXII
%%\DoubleSpacedXI

% If hyperref is used, dvi-to-ps driver of choice must be declared as
%   an additional option to the \documentclass. For example
%\documentclass[dvips,opre]{informs3}      % if dvips is used
%\documentclass[dvipsone,opre]{informs3}   % if dvipsone is used, etc.

%%% OPRE uses endnotes. If you do not use them, put a percent sign before
%%% the \theendnotes command. This template does show how to use them.
\usepackage{endnotes}
\let\footnote=\endnote
\let\enotesize=\normalsize
\def\notesname{Endnotes}%
\def\makeenmark{$^{\theenmark}$}
\def\enoteformat{\rightskip0pt\leftskip0pt\parindent=1.75em
  \leavevmode\llap{\theenmark.\enskip}}

% Private macros here (check that there is no clash with the style)

% Natbib setup for author-year style
\usepackage{natbib}
 \bibpunct[, ]{(}{)}{,}{a}{}{,}%
 \def\bibfont{\small}%
 \def\bibsep{\smallskipamount}%
 \def\bibhang{24pt}%
 \def\newblock{\ }%
 \def\BIBand{and}%

%% Setup of theorem styles. Outcomment only one.
%% Preferred default is the first option.
\TheoremsNumberedThrough     % Preferred (Theorem 1, Lemma 1, Theorem 2)
%\TheoremsNumberedByChapter  % (Theorem 1.1, Lema 1.1, Theorem 1.2)
\ECRepeatTheorems

%% Setup of the equation numbering system. Outcomment only one.
%% Preferred default is the first option.
\EquationsNumberedThrough    % Default: (1), (2), ...
%\EquationsNumberedBySection % (1.1), (1.2), ...

% In the reviewing and copyediting stage enter the manuscript number.
%\MANUSCRIPTNO{} % When the article is logged in and DOI assigned to it,
                 %   this manuscript number is no longer necessary

\usepackage{amsmath}
\usepackage{amssymb}
\usepackage{setspace}
\usepackage{paralist}
\usepackage{graphicx}
\usepackage{url}
\usepackage{algorithm}
\usepackage{algpseudocode}
\usepackage{multicol}
\usepackage{multirow}
\usepackage{csvsimple}

% Frequently used general mathematics
\newcommand{\R}{{\mathbb{R}}}
\newcommand{\Rp}{\R^+}
\newcommand{\Z}{{\mathbb{Z}}}
\newcommand{\Zp}{\Z^+}
\newcommand{\Q}{\mathbb{Q}}
\newcommand{\N}{\mathbb{N}}

% Commands for probability
\renewcommand{\P}{\mathbb{P}}
\newcommand{\E}{\mathbb{E}}
\newcommand{\p}[1]{\P \left[ #1 \right]}
\newcommand{\e}[1]{\E \left[ #1 \right]}
% \newcommand{\ee}[2]{\E_{#1} \left[ #2 \right]}

% Definitions of variables
\newcommand{\X}{X}
\newcommand{\x}{\mathbf{x}}
\newcommand{\xh}{\hat{\x}}
\newcommand{\lh}{\hat{\lambda}}
\newcommand{\mh}{\hat{\mu}}
\newcommand{\xs}{\x^*}
\newcommand{\xit}{\tilde{\mathbf{\xi}}}
\newcommand{\zt}{\tilde{z}}
\newcommand{\zs}{z^*}

% Further variables
\newcommand{\y}{\mathbf{y}}
\renewcommand{\c}{\mathbf{c}}
\newcommand{\A}{\mathbf{A}}
\renewcommand{\b}{\mathbf{b}}
\renewcommand{\k}{\mathbf{k}}
\newcommand{\D}{\mathbf{D}}
\newcommand{\B}{\mathbf{B}}
\renewcommand{\d}{\mathbf{d}}

% Epiconvergence for \plp
\newcommand{\qtrue}{q^{\text{true}}}

% Useful mathematics functions
\newcommand{\keywords}[1]{\par\noindent\enspace\ignorespaces\textbf{Keywords:} #1}
% \newcommand{\keywords}[1]{\par\addvspace\baselineskip\noindent\keywordname\enspace\ignorespaces #1}
% \DeclareMathOperator*{\argmin}{argmin}
% \theoremstyle{plain}
% \newtheorem{theorem}{Theorem}
% \newtheorem{lemma}[theorem]{Lemma}
% \newtheorem{proposition}[theorem]{Proposition}
% \newtheorem{corollary}[theorem]{Corollary}
% 
% \theoremstyle{definition}
% \newtheorem{definition}[theorem]{Definition}
% 
% \theoremstyle{remark}
% \newtheorem{remark}[theorem]{Remark}
\newtheorem{property}{Property}

\newcommand{\st}{\mbox{s.t.}}

% Naming shortcuts
\newcommand{\plp}{$\phi$LP-2}

\bibliographystyle{ormsv080}

%%%%%%%%%%%%%%%%
\begin{document}
%%%%%%%%%%%%%%%%

% Outcomment only when entries are known. Otherwise leave as is and 
%   default values will be used.
%\setcounter{page}{1}
%\VOLUME{00}%
%\NO{0}%
%\MONTH{Xxxxx}% (month or a similar seasonal id)
%\YEAR{0000}% e.g., 2005
%\FIRSTPAGE{000}%
%\LASTPAGE{000}%
%\SHORTYEAR{00}% shortened year (two-digit)
%\ISSUE{0000} %
%\LONGFIRSTPAGE{0001} %
%\DOI{10.1287/xxxx.0000.0000}%

% Author's names for the running heads
% Sample depending on the number of authors;
% \RUNAUTHOR{Jones}
% \RUNAUTHOR{Jones and Wilson}
% \RUNAUTHOR{Jones, Miller, and Wilson}
% \RUNAUTHOR{Jones et al.} % for four or more authors
% Enter authors following the given pattern:
\RUNAUTHOR{Love and Bayraksan}

% Title or shortened title suitable for running heads. Sample:
% \RUNTITLE{Bundling Information Goods of Decreasing Value}
% Enter the (shortened) title:
\RUNTITLE{A Data-Driven Method for Robust Water Allocation under Uncertainty}

% Full title. Sample:
% \TITLE{Bundling Information Goods of Decreasing Value}
% Enter the full title:
\TITLE{A Data-Driven Method for Robust Water Allocation under Uncertainty}

% Block of authors and their affiliations starts here:
% NOTE: Authors with same affiliation, if the order of authors allows, 
%   should be entered in ONE field, separated by a comma. 
%   \EMAIL field can be repeated if more than one author
\ARTICLEAUTHORS{%
\AUTHOR{David Love}
\AFF{University of Arizona, \EMAIL{dlove@email.arizona.edu}, \url{http://math.arizona.edu/~dlove/}}
\AUTHOR{G\"{u}zin~Bayraksan}
\AFF{The Ohio State University, \EMAIL{bayraksan.1@osu.edu}, \url{http://www-iwse.eng.ohio-state.edu/biosketch\_GBayraksan.cfm}}
% Enter all authors
} % end of the block

\ABSTRACT{%
	We investigate the application of techniques from distributionally robust optimization to water allocation in Tucson, AZ under future uncertainty.
	Tucson, like must of the southwestern United States, faces considerable uncertainty in its ability to provide water for its citizens in the future.
	The sources of this uncertainty include (1) the future population growth of the region, (2) the availability of water from the Colorado River, and (3) the effects of climate change and how it relates to water usage.
	This paper presents an approach for integrating forecasts for all these sources of uncertainty in a single optimization model, and provide insight on the value of constructing additional treatment facilities to mitigate the future risks.
}%

% Sample 
%\KEYWORDS{deterministic inventory theory; infinite linear programming duality; 
%  existence of optimal policies; semi-Markov decision process; cyclic schedule}

% Fill in data. If unknown, outcomment the field
\KEYWORDS{Data-driven optimization, water allocation, climate change}
%Optimization under uncertainty, water resources management,  ambiguous stochastic programming, robust optimization, environmental sustainability}
%\HISTORY{}
\maketitle
%%%%%%%%%%%%%%%%%%%%%%%%%%%%%%%%%%%%%%%%%%%%%%%%%%%%%%%%%%%%%%%%%%%%%%

% Samples of sectioning (and labeling) in IJOC
% NOTE: (1) \section and \subsection do NOT end with a period
%       (2) \subsubsection and lower need end punctuation
%       (3) capitalization is as shown (title style).
%
%\section{Introduction.}\label{intro} %%1.
%\subsection{Duality and the Classical EOQ Problem.}\label{class-EOQ} %% 1.1.
%\subsection{Outline.}\label{outline1} %% 1.2.
%\subsubsection{Cyclic Schedules for the General Deterministic SMDP.}
%  \label{cyclic-schedules} %% 1.2.1
%\section{Problem Description.}\label{problemdescription} %% 2.

\section{Introduction and Motivation}

The continued availability of water resources is in question in many parts of the world, especially in light of an uncertain climate future and population increase.
To combat these influences, methods of allocating water resources must adapt.
The Lower Colorado River Basin supplies water to 25 million people in California, Nevada and Arizona, making the management of this resource very important.
In Tucson, Arizona the Colorado River supplies more than 60\% of the water through the Central Arizona Project (CAP), the largest and most expensive aqueduct system in the United States.
This paper studies allocation of CAP water in a developing portion of Tucson under future climate and population uncertainty, creating a single integrated model using projections for each of these sources of uncertainty.

Climate change is one of the most important sources of uncertainty in evaluating the sustainability of water resources.
A wide variety of general circulation models (GCMs) have been developed for evaluating the effects of climate change.
The coarse results from these models are used as drivers for Regional Climate Models (RCMs) to generate temperature and precipitation predictions on finer scales.
To gain insight on still small regions, Bias Correction and Spatial Downscaling (BCSD) techniques have been applied to the results of the RCMs to generate these predictions on even finer scales \citep{bcsdreclamation}.
Methods of comparing and integrating the results of these differing models are important parts of water allocation planning.

In this paper, we adapt the $\phi$-divergence method of formulating two-stage stochastic linear programs from \cite{love2013phi}, which we call the two-stage $\phi$-divergence constrained ambiguous stochastic linear program with recourse (\plp), to a water allocation model.
The \plp\ model is an attractive approach to modeling uncertainty because it allows for using the data directly, integrating those scenarios of interest into a single modeling formulation.
These scenarios can come from direct observation, results of simulation, or from expert opinion that the decision maker would especially like to be robust against.
In this water allocation problem, scenarios are generated from projections of population growth, CAP water supply, and climate-driven per-capita demand for water resources.

\subsection{Related Literature}

Stochastic programming methods have a long history in water resources management.
One early example is \cite{hall1970optimal}, which formulates a stochastic linear programming model for water rights allocation in arid and semiarid agriculture.
\cite{bishop1975optimization} provides an early stochastic model of a water allocation system, including wastewater treatment and reuse.
More recent work includes \cite{higgins2008stochastic}, which provides a multistage nonlinear stochastic formulation of water resource allocation and applies it to South East Queensland in Australia.
Other studies have used interval-stochastic programming to formulate the model, e.g., \cite{li2009multistage}.
\cite{shao2011conditional} combine inexact stochastic programming with uncertainty measured by the Conditional Value at Risk (CVaR).
This is particularly similar to the work presented here, because CVaR can be represented as a $\phi$-divergence \citep{love2013phi}.

Another optimization model for water distribution developed by \cite{wang2008basin} considers the entire South Saskatchewan River Basin.
This study applies two multi-objective optimization models to generate water management solutions that are both equitable and economically efficient using the Cooperative Water Allocation Model.

Most similar to the water allocation model considered here is the CALVIN (CALifornia Value Integrated Network) model that has been used to study water distribution in California \citep{jenkins2004optimization}.
CALVIN is a multi-period generalized network flow model developed to minimize the total economic cost of water distribution in California.
Much of the work using CALVIN has been aimed towards analyzing the impact of climate changes.
\cite{harou2010economic} integrated sustained drought scenarios into the CALVIN model based on historical records of long-term droughts in California.
Later, \cite{connell2011adapting} integrate the GFDL-CM3 climate model, one of the climate models used in this work, into CALVIN as a method of evaluating the impact of a warmer and dryer climate on water availability in the future.

Several studies have been directed towards evaluating the relative benefits of the GCMs and associated RCMs.
NARCCAP, the North American Regional Climate Change Assessment Program, compares several models by showing how they perform using historical observations, and computing climate scenarios in the 21st century.
\cite{mearns2012north} compare the results of six RCMs using observational data over much of the United States and Canada from 1980-2004.
\cite{brekke2008significance} study 17 climate models from the World Climate Research Programme's (WCRP's) Coupled Model Intercomparison Project
phase 3 (CMIP3), and determine model credibility criteria for Northern California.
This comparison helped to inform our choices of climate models included in the present work.

\subsection{Contributions}

This paper applies the \plp\ method described in \cite{love2013phi} to a model for water treatment and distribution in a developing region of Tucson, AZ described in \cite{woods2012centralized}.
This work integrates multiple sources of uncertainty into a single optimization model to estimate the relative risk posed by each unknown quantity.
These sources of uncertainty include: (1) population growth in the region considered, (2) Tucson's water allocation from the Lower Colorado River Basin, and (3) the climate of the region and its effect on per-capita demand for water in the region.
We generate scenarios for each of these variables, and integrate them into a single \plp\ stochastic programming model, and estimate the long-run value of constructing additional water treatment facilities to decrease the severity of future water shortages.

\subsection{Organization}

The rest of the paper is organized as follows.
Section \ref{sec:network_model} presents the water distribution model and the planning scenarios used.
Section \ref{sec:phi_divergences} briefly describes the \plp\ approach from \cite{love2013phi} and reviews some of the key features.
Section \ref{sec:water} describes the application to Tucson, AZ, and Section \ref{sec:comp} presents the results of solving this application with the \plp\ method.
We end in Section \ref{sec:concl} with our conclusions.



%%%%%%%%%%%%%%%%%%%%%%%%%%%%%%%%%%%%%%%%%%%%%%%%%%%%%%%%%%%%%%%%%%%%%%%%%%%%%%%%
\section{Generalized Network Water Model}
\label{sec:network_model}

A generalized network model of water distribution is defined by a set of nodes and directed arcs $(N,\: A)$.
The nodes represent available water supply inputs, water treatment plants, reservoirs, interconnection points, and water demand sites.
The arcs represent the conveyance system (pipes, etc.) that carry water between the nodes.
Water can be stored in between time periods in reservoirs to meet future demands.
The model aims to find the minimal cost water flows considering energy, treatment, storage, and transportation costs over the planning period. 
Generalized network water allocation models have been used to find water allocations and delivery reliabilities and to assess values of different water use operations; see, e.g., \cite{draper_etal_03}. 

Water flows on arc $(i,j) \in A$ during time period $t = 1, \dots, P$ are represented by decisions $x_{ijt}$.
Each arc $(i,j) \in A$ and time period $t$ has a unit cost $c_{ijt}^x$, loss coefficient $0 \leq a_{ijt} \leq 1$ to account for evaporation, leakage from the pipes, etc., and bounds on the flow $l_{ijt}^x \leq x_{ijt} \leq u_{ijt}^x$.
Each node $j \in N$ has a supply/demand for time period $t$, $b_{jt}$.
Nodes representing reservoirs are able to store water between time periods.
Nodes capable of storing water are denoted as members of the set $S$.
Stored water available at node $j$ at the end of time period $t$ is $s_{jt}$, with associated cost $c_{jt}^s$ and bounds $l_{jt}^s \leq s_{jt} \leq u_{jt}^s$.
Finally, water released into the environment from node $j$ in period $t$ is given by $r_{jt}$, with bounds $l_{jt}^r \leq r_{jt} \leq u_{jt}^r$.
The deterministic model is a multi-period generalized network flow model of the form
\begin{align*}
	\min_{x,s,r} \ & \sum_{(i,j) \in A} \sum_{t=1}^P c_{ijt}^x x_{ijt} + \sum_{j \in N} \sum_{t=1}^P c_{jt}^s s_{jt} \\
	\st \ & \sum_{j : (j,i) \in A} a_{jit} x_{jit} = \sum_{j : (i,j) \in A} x_{ijt} + r_{jt} + b_{jt} \ \ \forall i \not\in S, t \\
	& \sum_{j : (j,i) \in A} a_{ji,t-1} x_{ji,t-1} + s_{j,t-1} = \sum_{j : (i,j) \in A} x_{ijt} + s_{jt}, \ \ \forall i \in S,t \\
	& l_{ijt}^x \leq x_{ijt} \leq u_{ijt}^x,\ \ \ \forall i,j,t \\
	& l_{jt}^s \leq s_{jt} \leq u_{sjt}^s, \ \ \ \forall j,t \\
	& l_{jt}^r \leq r_{jt} \leq u_{sjt}^r, \ \ \ \forall j,t.
\end{align*}
To compute the first-year constraints, we assume that $x_{ij0} \equiv 0$ and that $s_{j0}$ is given.

The model is converted to a two-stage stochastic model with $P_1$ periods in the first stage and $P-P_1$ stages in the second stage.
We assume that the supplies and demands are uncertain, as well as the bounds on the decision variables.
\begin{align}
	\min_{(x,s,r) \in L^1} \ & \sum_{(i,j) \in A} \sum_{t=1}^{P_1} c_{ijt}^x x_{ijt} + \sum_{j \in N} \sum_{t=1}^{P_1} c_{jt}^s s_{jt} + \sum_{\omega=1}^n q_\omega h_\omega(s) \label{eq:gen_network_two_stage} \\
	\st \ & \sum_{j : (j,i) \in A} a_{jit} x_{jit} = \sum_{j : (i,j) \in A} x_{ijt} + r_{jt} + b_{jt} \ \ \forall i \not\in S, 1 \leq t \leq P_1 \notag \\
	& \sum_{j : (j,i) \in A} a_{ji,t-1} x_{ji,t-1} + s_{j,t-1} = \sum_{j : (i,j) \in A} x_{ijt} + s_{jt},\ \ \ \ \forall i \in S, 1 \leq t \leq P_1, \notag
\end{align}
where $\{q_\omega\}_{\omega = 1}^n$ is the probability distribution of the uncertain second stage problem $h_\omega(s)$, given by
\begin{align}
	h_\omega(s) = \min_{(x,s,r) \in L^2_\omega} \ & \sum_{(i,j) \in A} \sum_{t=P_1+1}^{P} c_{ijt}^x x_{ijt} + \sum_{j \in N} \sum_{t=P_1+1}^{P} c_{jt}^s s_{jt} \label{eq:gen_network_second_stage} \\
	\st \ & \sum_{j : (j,i) \in A} a_{jit} x_{jit}^\omega = \sum_{j : (i,j) \in A} x_{ijt}^\omega + r_{jt}^\omega + b_{jt}^\omega, \ \ \ \ \forall j, P_1+1 \leq t \leq P, \notag \\
	& \sum_{j : (j,i) \in A} a_{ji,t-1} x_{ji,t-1} + s_{j,t-1} = \sum_{j : (i,j) \in A} x_{ijt}^\omega + s_{jt}^\omega,\ \ \ \ \forall i \in S, t = P+1, \notag \\
	& \sum_{j : (j,i) \in A} a_{ji,t-1} x_{ji,t-1}^\omega + s_{j,t-1}^\omega = \sum_{j : (i,j) \in A} x_{ijt}^\omega + s_{jt}^\omega,\ \ \ \ \forall i \in S, P_1+1 \leq t \leq P, \notag
\end{align}
and $L^1$ and $L^2_\omega$ represent the feasible regions defined by the lower and upper variable bounds.



%%%%%%%%%%%%%%%%%%%%%%%%%%%%%%%%%%%%%%%%%%%%%%%%%%%%%%%%%%%%%%%%%%%%%%%%%%%%%%%%
\section{The \plp\ Problem}
\label{sec:phi_divergences}

We now turn our attention to formulating the \plp\ model.
We first define the concept of a $\phi$-divergence and describe some of it's properties.
\cite{pardo2005statistical} provides a good overview of much of the known properties of $\phi$-divergences.
After defining the \plp\ problem, we review some of the properties discussed in \cite{love2013phi}, and other properties that can be found in \cite{bental2011robust}.

$\phi$-divergences are used in statistics to measure the ``distance'' between two distributions. 
In the discrete case, $\phi$-divergences can be used generally to measure the distance between two probability vectors $p = (p_1, \dots, p_n)^T$ and $q = (q_1, \dots, q_n)^T$, i.e., vectors satisfying $\sum_{\omega=1}^n p_\omega = \sum_{\omega=1}^n q_\omega = 1$.
The $\phi$-divergence is defined by
\[
	I_\phi(p,q) = \sum_{\omega=1}^n q_\omega \phi\left(\frac{p_\omega}{q_\omega}\right),
\]
where $\phi(t)$, called the $\phi$-divergence function, is a convex function on $t \geq 0$ such that $\phi(t) \geq 0$ and $\phi(1) = 0$, and with the additional interpretations that $0 \phi(a/0) = a \lim_{t \rightarrow \infty} \frac{\phi(t)}{t}$, and $0 \phi(0/0) = 0$

$\phi$-divergences are not, in general, metrics.
For example, most $\phi$-divergences do not satisfy the triangle inequality and many are not symmetric in the sense that $I_\phi(p,q) \neq I_\phi(q,p)$.
One exception is the Variation distance, which is equivalent to the $L^1$-distance between the vectors.

A $\phi$-divergence has an adjoint, defined by
\begin{equation} \label{eq:adjoint}
	\tilde{\phi}(t) = t \phi\left(\frac{1}{t}\right),
\end{equation}
which satisfies all criteria for a $\phi$-divergence \citep{bental1991certainty}, and has the property that $I_{\tilde{\phi}}(p,q) = I_\phi(q,p)$.
Divergences that are symmetric with respect to the input vectors are known as self-adjoint.

The problem formulation involves use of the conjugate $\phi^* : \R \rightarrow \R \cup \{\infty\}$, defined as
\begin{equation} \label{eq:conjugate}
	\phi^*(s) = \sup_{t \geq 0} \{st - \phi(t)\}.
\end{equation}
The conjugate $\phi^*$ is a nondecreasing convex function and may be undefined above some upper bound $\bar{s}$.

Table \ref{tb:phi_definitions} lists some common examples of $\phi$-divergences, along with their adjoints and conjugates.
For all divergences, $\phi(t) = \infty$ for $t < 0$, and the value of the conjugate is listed only in its domain; i.e., $\{s : \phi^*(s) < \infty\}$.

\begin{table}
	\TABLE
	{
		Definitions of some common $\phi$-divergences, along with their adjoints $\tilde{\phi}(t)$ and conjugates $\phi^*(s)$
		\label{tb:phi_definitions}
	}
	{\begin{tabular}{lccccc}
		\hline \\
		Divergence                        & $\phi(t)$          & $\tilde{\phi}(t)$               & $\phi(t), t \geq 0$   & $I_\phi(p,q)$     & $\phi^*(s)$ \\
		\hline
		Kullback-Leibler                 & $\phi_{kl}$        & $\phi_b$                        & $t\log t - t + 1$     & $\sum p_\omega \log\left(\frac{p_\omega}{q_\omega}\right)$ & $e^s - 1$ \\
		Burg Entropy                      & $\phi_b$           & $\phi_{kl}$                     & $-\log t + t - 1$     & $\sum q_\omega \log\left(\frac{q_\omega}{p_\omega}\right)$ & $-\log(1-s),\ s < 1$  \\
		J-Divergence                      & $\phi_j$           & $\phi_j$                        & $(t-1)\log t$         & $\sum (p_\omega - q_\omega) \log\left(\frac{p_\omega}{q_\omega}\right)$ & No closed form \\
		$\chi^2$-Distance                 & $\phi_{\chi^2}$    & $\phi_{m\chi^2}$                & $\frac{1}{t} (t-1)^2$ & $\sum \frac{(p_\omega-q_\omega)^2}{p_\omega}$              & $2 - 2\sqrt{1-s},\ s < 1$  \\
		Modified $\chi^2$-Dist.           & $\phi_{m\chi^2}$   & $\phi_{\chi^2}$                 & $(t-1)^2$             & $\sum \frac{(p_\omega - q_\omega)^2}{q_\omega}$            & $\begin{cases} -1 & s < -2 \\ s + \frac{s^2}{4} & s \geq -2 \end{cases}$ \\
% 		$\chi$-div,  $\theta > 1$ & $\phi_\chi^\theta$ & $t^{1-\theta}\phi_\chi^\theta$ & $|t-1|^\theta$         & $\sum q_\omega |1-\frac{p_\omega}{q_\omega}|^\theta$       & $\begin{cases} -1 & s \leq -\theta \\ s + (\theta-1)\left(\frac{|s|}{\theta}\right)^\frac{\theta}{\theta-1}  & s \geq -\theta \end{cases}$ \\
		Variation Distance                & $\phi_v$           & $\phi_v$                        & $|t-1|$               & $\sum |p_\omega - q_\omega|$                               & $\begin{cases} -1 & s \leq -1 \\ s & -1 \leq s \leq 1 \end{cases}$ \\
		Hellinger Distance                & $\phi_h$           & $\phi_h$                        & $(\sqrt{t} - 1)^2$    & $\sum (\sqrt{p_\omega} - \sqrt{q_\omega})^2$               & $\frac{s}{1-s},\ s < 1$ \\
	\hline
	\end{tabular}}
	{}
\end{table}

%%%%%%%%%%%%%%%%%%%%%%%%%%%%%%%%%%%%%%%%%%%%%%%%%%%%%%%%%%%%%%%%%
\subsection{Formulating \plp}
\label{ssec:form}

In this section, we simplify the notation for the first-stage (\ref{eq:gen_network_two_stage}) and second-stage (\ref{eq:gen_network_second_stage}) problems as follows.
In the first stage, decision variables $\{x_{ijt}\}$, $\{s_{jt}\}$ and $\{r_{jt}\}$ become the vector $\x$, costs $\{c_{ijt}^x\}$ and $\{c_{jt}^s\}$ are written as the row vector $\c$, the supply/demand parameters $b_{jt}$ become the vector $\b$ and the constraint matrix is written as $A$.
In the second stage, we denote the decisions as $\y^\omega$, the costs as $\k^\omega$, the supply/demands as $\d^\omega$, and the constraint matrices multiplying $\y^\omega$ and $\x$ as $D^\omega$ and $B^\omega$, respectively.

We begin with a two-stage stochastic linear program with recourse (SLP-2).
Let $\x$ be a vector of first-stage decision variables with cost vector $\c$, constraint matrix $\A$ and right-hand side $\b$.
We assume a finite distribution given by $q_\omega$ with scenarios indexed by $\omega = 1, \dots, n$.
The SLP-2 is
\begin{align}
	\min_\x \ & \left\{ \c\x + \sum_{\omega=1}^n q_\omega h_\omega(\x) : \A\x = \b, \x \geq 0 \right\}, \label{eq:slp_first_stage}
\end{align}
where
\begin{align}
	h_\omega(\x) = \min_{\y^\omega} \ & \left\{ \k \y^\omega : \D \y^\omega = \B \x + \d^\omega, \y^\omega \geq 0 \right\}. \label{eq:slp_second_stage}
\end{align}
We assume relatively complete recourse; i.e., the second-stage problems $h_\omega(\x)$ are feasible for every feasible solution $\x$ of the first-stage problem; and that the second-stage problems $h_\omega(\x)$ are dual feasible for every feasible solution $\x$ of the first-stage problem.
For convenience, we denote $X = \{\x : \A\x = \b, \x \geq 0\}$.

The SLP-2 formulation assumes that the distribution $\{q_\omega\}_{\omega=1}^n$ is known.
However, in many applications the distribution is itself unknown.
One technique to deal with this is to replace the known distribution with an {\it ambiguity set} of distributions; i.e., a set of distributions that is believed to contain the true distribution.
In this paper, we construct the ambiguity set by considering all distributions whose $\phi$-divergence from the nominal distribution $q$ is sufficiently small.
We focus on a data-driven setting and assume that $q$ is generated from observations, where scenario $\omega$ has been observed $N_\omega$ times, with $N = \sum_{\omega=1}^n N_\omega$ total observations, although $q$ can be obtained in other ways.
In SLP-2, this data-driven setting would correspond to the probability of scenario $\omega$ to be $q_\omega = \frac{N_\omega}{N}$.

By replacing the specific distribution in SLP-2 with a set of distributions sufficiently close to the nominal distribution with respect to $\phi$-divergence, we create the \plp\ model.
In the \plp, the objective function is minimized with respect to the worst-case distribution selected from the ambiguity set of distributions.
The resulting minimax formulation of \plp\ is
\begin{equation}
	\min_{\x \in X} \max_{p \in \mathcal{P}} \left\{ \c\x + \sum_{\omega=1}^{n} p_\omega h_\omega(\x) \right\}, \label{eq:plp_primal}
\end{equation}
where the ambiguity set is
\begin{align}
	\mathcal{P} = & \left\{ \sum_{\omega = 1}^{n} q_\omega \phi\left(\frac{p_\omega}{q_\omega}\right) \leq \rho, \right. \label{eq:plp_primal_divergence} \\
	& \ \sum_{\omega=1}^{n} p_\omega = 1, \label{eq:plp_primal_probability} \\
	& \ \left. p_\omega \geq 0,\ \forall \omega \right\}. \label{eq:nonneg}
\end{align}
We refer to (\ref{eq:plp_primal_divergence}) as the $\phi$-divergence constraint and (\ref{eq:plp_primal_probability}) and (\ref{eq:nonneg}) simply ensure a probability measure.
We discus how to determine $\rho$ in (\ref{eq:plp_primal_divergence}) in Section \ref{ssec:robust_level}.

Taking the dual of the inner maximization problem, with dual variables $\lambda$ and $\mu$, of constraints (\ref{eq:plp_primal_divergence}) and (\ref{eq:plp_primal_probability}), respectively, and combining the two minimizations gives \plp\ in dual form
\begin{align}
	\min_{\x,\lambda,\mu} \ & \c\x + \mu + \rho \lambda + \lambda \sum_{\omega=1}^{n} q_\omega \phi^*\left(\frac{h_\omega(\x) - \mu}{\lambda}\right) \label{eq:plp_two_stage} \\
	\st \ & \x \in X \nonumber \\
	& \frac{h_\omega(\x) - \mu}{\lambda} \leq \lim_{t \rightarrow \infty} \frac{\phi(t)}{t}, \ \forall \omega \label{eq:plp_feas_constraint}\\
	& \lambda \geq 0, \nonumber
\end{align}
where $h_\omega(\x)$ and the second-stage problems are as given in (\ref{eq:slp_second_stage}), $0\phi^*(s/0)=0$ if $s\leq 0$ and  $0\phi^*(s/0)=+\infty$ if $s > 0$.
Note that some $\phi$, such as the J-Divergence, have no closed form representation of $\phi^*$, but can be expressed as the sum of other divergences---Burg Entropy and KL divergence---which allows the dual to be formed; see \citep{bental2011robust} for details.
Theorem 1 of \cite{bental2011robust} contains a derivation of the dual problem.
Note in particular that the dual formulation is accurate even for $q_\omega = 0$ for some $\omega$.
Also note that the right-hand side of (\ref{eq:plp_feas_constraint}) contains a limit.
For some $\phi$-divergences, like the KL divergence, this limit is $\infty$, in which case (\ref{eq:plp_feas_constraint}) is redundant.
However, other $\phi$-divergences, like the Hellinger distance, have a finite limit, inducing this constraint.
Throughout the paper, we use $s_\omega$ to denote
\begin{equation}
	s_\omega = \frac{h_\omega(\x) - \mu}{\lambda}. \label{eq:s_omega_definition}
\end{equation}


%%%%%%%%%%%%%%%%%%%%%%%%%%%%%%%%%%%%%%%%%%%%%%%%%%%%%%%%%%%%%%%%%
\subsection{Basic Properties}
\label{ssec:basicprop}

In this section we list some basic properties of \plp.
Most of these have already been noted earlier (e.g., by \cite{bental2011robust} and others for specific $\phi$-divergences) but we list them for completeness.
Some of these properties help with our specialized solution method and we refer to them in later sections.

\begin{property}
	\label{property:convex}
	\plp\ is a convex program.
\end{property}

\begin{property}
	\label{property:coherent_risk_measure}
	\plp\ is equivalent to minimizing a coherent risk measure.
\end{property}

Note that being a coherent risk measure implies that \plp\ is a convex problem.

\begin{property}
	\label{property:time_structure}
	\plp\ preserves the time structure of SLP-2.
\end{property}

\begin{property}
	\label{property:primal_dual_relation}
	The worst-case distribution can be calculated with the equations
	\begin{equation}\label{eq:p_worst}
		\frac{p_\omega}{q_\omega} \in \partial \phi^*\left(s_\omega\right), \ \ \ \ \ \sum_{\omega=1}^n q_\omega \phi\left(\frac{p_\omega}{q_\omega}\right) \leq \rho, \ \ \ \ \ \sum_{\omega=1}^n p_\omega = 1.
	\end{equation}
\end{property}

We now discuss these properties. A coherent risk measure is defined in \citep{rockafellar2007coherent}, which shows that minimizing a coherent risk measure over a polyhedron implies that \plp\ is a convex problem.
The convexity of \plp\ was also noted in \citep{bental2011robust}.
Properties \ref{property:time_structure} and \ref{property:primal_dual_relation} help with our decomposition-based solution method described in \cite{love2013phi}.
The preservation of time structure, as can be seen in (\ref{eq:plp_two_stage}), allows us to decompose the problem and convert (sub-)derivatives of $h_\omega(\x)$ to (sub-)derivatives of $\phi^*\left(s_\omega\right)$, aiding in our decomposition-based solution method. 
The appearance of the conjugate $\phi^*(s)$ in the objective of (\ref{eq:plp_two_stage}) gives a method for retrieving the worst-case distribution from the dual problem, as detailed in Property \ref{property:primal_dual_relation}.
In many cases, the first equation in (\ref{eq:p_worst}) is sufficient to calculate $\{p_\omega\}_{\omega=1}^n$.
In addition, $\phi^*$ is often differentiable and so we have the relationship $p_\omega = q_\omega \phi^{* \prime}(s_\omega)$.
Special cases when $\lambda = 0$ or $q_\omega = 0$ for some $\omega$ are detailed in Section \ref{ssec:classification}.


%%%%%%%%%%%%%%%%%%%%%%%%%%%%%%%%%%%%%%%%%%%%%%%%%%%%%%%%%%%%%%%%%
\subsection{The Level of Robustness}
\label{ssec:robust_level}

The literature on $\phi$-divergences provides some insight on choosing a reasonable asymptotic value of $\rho$ in the data-driven setting. 
When $\phi$ is twice continuously differentiable around $1$ with $\phi^{\prime \prime}(1)>0$, Theorem 3.1 of \cite{pardo2005statistical} shows that the statistic $T^\phi_N(q^N,\qtrue) = \frac{2N}{\phi''(1)} \sum_{\omega=1}^n \qtrue_\omega \phi\left(\frac{q^N_\omega}{\qtrue_\omega}\right)$ converges in distribution to a $\chi^2$-distribution with $n-1$ degrees of freedom, where $q^N$ denotes the empirical distribution ($q^N_\omega = N_\omega/N$), and $\qtrue$ denotes the underlying true distribution.
Most $\phi$-divergences in Table~\ref{tb:phi_definitions} satisfy this differentiability condition.
\cite{bental2011robust} then use this result to suggest the asymptotic value
\begin{equation} \label{eq:asymptotic_rho}
	\rho = \frac{\phi''(1)}{2N} \chi^2_{n-1,1-\alpha},
\end{equation}
where $\chi^2_{n-1,1-\alpha}$ is the $1-\alpha$ percentile of a $\chi^2_{n-1}$ distribution, which produces an approximate $1-\alpha$ confidence region on the true distribution.
For corrections for small sample sizes and more details, we refer the readers to \citep{pardo2005statistical} and \citep{bental2011robust}.

%%%%%%%%%%%%%%%%%%%%%%%%%%%%%%%%%%%%%%%%%%%%%%%%%%%%%%%%%%%%%%%
\subsection{A Classification of $\phi$-Divergences}
\label{ssec:classification}

Given that there are many $\phi$-divergences to choose from, it is important to study how $\phi$-divergences act within an ambiguous (or, distributionally robust) stochastic optimization model. 
We present a classification of $\phi$-divergences into four types, resulting from an examination of the limiting behavior of $\phi(t)$ as $t \rightarrow 0$ and $t \rightarrow \infty$.
Different classifications may be suitable to different problem types and desired qualities in the ambiguous model---we discuss modeling considerations with respect to our classification in Section \ref{sssec:suppressandpop}.

%%%%%%%%%%%%%%%%%%%%%%%%%%%%%%%%%%%%%%%%%%%%%%%%%%%%%%%%%%%%%%%
\subsubsection{Suppressing and Popping of Scenarios}
\label{sssec:suppressandpop}

As motivation for our classification, consider a self-adjoint $\phi$-divergence, which satisfies the relation
\begin{equation} \label{eq:self_adjoint_classification}
	\frac{\phi(t)}{t} = \phi\left(\frac{1}{t}\right),
\end{equation}
and consider $t \rightarrow \infty$.
If both sides of (\ref{eq:self_adjoint_classification}) are finite in the limit, then we see a correspondence between the boundedness of $\phi(t)$ for $t < 1$ and linear growth of $\phi(t)$ for $t > 1$.
On the other hand, infinite limits of (\ref{eq:self_adjoint_classification}) indicate a correspondence between superlinear growth of $\phi(t)$ for $t > 1$ and unboundedness of $\phi(t)$ for $t < 1$.

Recall the definition of the ambiguity set, in particular, the $\phi$-divergence constraint (\ref{eq:plp_primal_divergence}). 
In the \plp, $\phi$ has arguments given by ratios of probabilities, $\tfrac{p_\omega}{q_\omega}$ and the limits $t \rightarrow 0$ and $t \rightarrow \infty$ correspond to the cases when $p_\omega = 0$ and $q_\omega = 0$, respectively.
Consider each of these limiting cases:
\begin{itemize}
	\item {\sc Case 1:} $q_\omega > 0$ but $p_\omega = 0$.
		We call this the ``{\bf Suppress}'' behavior because a scenario with a positive probability in the nominal distribution can take zero probability in the ambiguous problem. In this case we need to examine $\lim_{t \searrow 0} \phi(t)$:
	\begin{itemize}
		\item If $\lim_{t \searrow 0} \phi(t) = \infty$, the ambiguity region will never contain distributions with $p_\omega = 0$ but $q_\omega > 0$.
		\item On the other hand, if $\lim_{t \searrow 0} \phi(t) < \infty$, the ambiguity region could contain such a distribution, provided $q_\omega$ is sufficiently small or $\rho$ is sufficiently large.
			We say that such a $\phi$-divergence can \emph{suppress} scenario $\omega$.
	\end{itemize}
	\item {\sc Case 2:} $q_\omega = 0$ but $p_\omega > 0$.
		We call this the ``{\bf Pop}'' behavior because a scenario with zero probability in the nominal distribution can have a positive probability (or, pop) in the ambiguous problem. In this case, we need to examine $\lim_{t \nearrow 0} \frac{\phi(t)}{t}$:
	\begin{itemize}
		\item If $\lim_{t \nearrow 0} \frac{\phi(t)}{t} = \infty$, the ambiguity region can never contain distributions with $p_\omega > 0$ but $q_\omega = 0$.
		\item On the other hand, if $\lim_{t \nearrow 0} \frac{\phi(t)}{t} < \infty$, the ambiguity region will admit sufficiently small $p_\omega$.
			We say that these $\phi$-divergences can \emph{pop} scenario $\omega$.
	\end{itemize}
	\item {\sc Case 3:} $p_\omega = 0$ but $q_\omega = 0$.
		Such a situation has no contribution to the divergence, since $0 \phi\left(\tfrac{0}{0}\right) = 0$.
\end{itemize}

\noindent These two limiting cases describing suppressing and popping behavior in $\phi$-divergences create four distinct categories.
Examples of divergences in each category are given in Table \ref{tb:phi_categories}.
Note that $\phi$ can suppress scenarios if and only if its adjoint $\tilde{\phi}$ can pop scenarios, and vice versa.
This means that self-adjoint $\phi$-divergences are either capable of both popping and suppressing scenarios or capable of neither.

\begin{table}
	\TABLE
	{
		Examples of $\phi$-divergences fitting into each category.
		The number in parentheses under the ``Can Suppress Scenarios'' column denotes the subcategory detailed in \cite{love2013phi}.
		\label{tb:phi_categories}
	}
	{\begin{tabular}{l|p{.33\textwidth}p{.33\textwidth}}
		 & Can Suppress Scenarios & Cannot Suppress Scenarios \\
		 \hline
		 Can Pop Scenarios %
			& \parbox{.33\textwidth}{Hellinger Distance (2),\\Variation Distance (1)} %
			& \parbox{.33\textwidth}{Burg Entropy,\\$\chi^2$-Distance} \smallskip \\
		 Cannot Pop Scenarios %
			& \parbox{.33\textwidth}{Kullback-Leibler Divergence (2),\\Modified $\chi^2$-Distance (1)} %
			& \parbox{.33\textwidth}{J-Divergence}
	\end{tabular}}
	{}
\end{table}

%%%%%%%%%%%%%%%%%%%%%%%%%%%%%%%%%%%%%%%%%%%%%%%%%%%%%%%%%%%%%%%
\subsubsection{Modeling Considerations When Choosing a Divergence}
\label{sssec:model}

We offer the following suggestions for choosing an appropriate $\phi$-divergence classification for the data available.
First, consider whether to choose a distribution that can suppress scenarios.
If the problem scenarios come from high-quality observed data, one may wish to avoid divergences that can suppress scenarios.
However, if the data is poorly sampled or comes from opinion rather than observation or simulation, the option of suppressing scenarios may result in a solution with better robustness properties.

Next, consider whether to choose a distribution that allows for popping scenarios.
If the problem scenarios come strictly from observation, with little theoretical understanding of the problem, we suggest choosing a divergence that cannot pop scenarios.
However, if the problem scenarios come from a mix of observed/simulated data and expert opinion about scenarios of interest, then divergences that can pop present an interesting modeling choice.
This allows for including interesting but unobserved scenarios, allowing the mathematical program to assign an appropriate probability to them.



%%%%%%%%%%%%%%%%%%%%%%%%%%%%%%%%%%%%%%%%%%%%%%%%%%%%%%%%%%%%%%%%%%%%%%%%%%%%%%%%
\subsection{Value of Data}
\label{ssec:value}

In this section we assume the nominal distribution $q$ is the empirical distribution ($q_\omega = \tfrac{N_\omega}{N}$) and provide insight into how the \plp\ changes when a single observation.
This analysis must consider how $\rho$ changes as additional samples are taken; therefore, we use $\rho_N$ to emphasize the dependence on sample size in this section.
To be consistent with the known $\phi$-divergence results stated in Section \ref{ssec:robust_level}, we assume $\rho_N = \frac{\rho_0}{N}$.

With a data-driven formulation such as \plp, it is natural to ask how the optimal value and solution changes as more data is gathered.
In particular, one might be concerned about being overly conservative in the problem formulation and thus missing the opportunity to find a better solution to the true distribution.
For \plp, this means that the initial model is likely to be more conservative in an effort to be robust, while the new information could make the model less conservative because new information removes the current worst-case distribution from the ambiguity set.  
Below, we present a simple method of determining if taking an additional sample will eliminate the old worst-case distribution and allow for better optimization; i.e., a lower-cost solution.
%We also provide a way of estimating the probability of sampling such an observation.

\begin{theorem}[Theorem 1 from \cite{love2013phi}]
	\label{thm:value}
	An additional sample of scenario $\hat{\omega}$ will result in a decrease in the worst-case expected cost of the \plp\ if the following condition is satisfied
	\begin{equation} \label{eq:cost_decrease_cond}
		\sum_{\omega=1}^n q_\omega \phi^{*\prime}\left(\frac{N}{N+1}s^*_\omega\right) \left(\frac{N}{N+1}s^*_\omega\right) > \phi^*\left(\frac{N}{N+1}s^*_{\hat{\omega}}\right),
	\end{equation}
	where $s^*_\omega = \dfrac{h_\omega(\x^*_N) - \mu^*_N}{\lambda^*_N}$ and $(\x^*_N,\mu^*_N,\lambda^*_N)$ solve the $N$-sample problem with $q_\omega = \tfrac{N_\omega}{N}$.
\end{theorem}

We can interpret \eqref{eq:cost_decrease_cond} as follows. If an additional sample is taken from the unknown distribution and the resulting observed scenario $\hat{\omega}$ satisfies (\ref{eq:cost_decrease_cond}), then the $(N+1)$-sample problem will have a lower cost than the $N$-sample problem that was already solved.
This is equivalent to saying that an additional observation of $\hat{\omega}$ will rule out the computed worst-case distribution given by $\{p_\omega\}_{\omega=1}^{n}$ in \eqref{eq:p_worst}.

It is possible to simplify the condition in \eqref{eq:cost_decrease_cond} for some $\phi$-divergences and we detail this in the corollary below. 

\begin{corollary}[Corollary 1 from \cite{love2013phi}]
	\label{cor:cost_decrease_trick}
	An additional sample of scenario $\hat{\omega}$ will result in a decrease in the worst-case expected cost of the \plp\ if the following condition is satisfied for:\vspace*{-0.1in}
	\begin{multicols}{2}
		\begin{description}
			\item[Burg entropy:] $\frac{p_{\hat{\omega}}}{q_{\hat{\omega}}} < \frac{N}{N+1}$, %(or, $p_{\hat{\omega}}<\frac{N_{\hat{\omega}}}{N}$)
			\item[$\chi^2$-distance:]  $\sum_\omega q_\omega \frac{q_\omega}{p_\omega} + \sqrt{\frac{N+1}{N}} < 2 \frac{p_{\hat{\omega}}}{q_{\hat{\omega}}}$,
			\item[Hellinger:] $\sum_\omega q_\omega \sqrt{\frac{p_\omega}{q_\omega}} + \sqrt{\frac{p_{\hat{\omega}}}{q_{\hat{\omega}}}} < 2 \frac{N}{N+1}$,
			\item[Modified $\chi^2$:] $2 \sum_\omega p_\omega \frac{p_\omega}{q_\omega} > \left(\frac{p_{\hat{\omega}}}{q_{\hat{\omega}}}\right)^2 + \left(\frac{N+1}{N}\right)^2$.
		\end{description}
	\end{multicols}
\end{corollary}

The simple conditions in Theorem~\ref{thm:value} and Corollary~\ref{cor:cost_decrease_trick} provide insight into different scenarios for a decision maker. 
Let $L = \left\{ \hat{\omega} : \sum_{\omega=1}^n q_\omega \phi^{*\prime}\left(\frac{N}{N+1}s^*_\omega\right) \left(\frac{N}{N+1}s^*_\omega\right) > \phi^*\left(\frac{N}{N+1}s^*_{\hat{\omega}}\right) \right\}$.
Set $L$ divides the scenarios into two---the ones in $L$ guarantee a drop in the overall cost if sampled one more and therefore can be considered ``good'' scenarios. 
Note that scenarios not in $L$ can also result in the cost decrease.



%%%%%%%%%%%%%%%%%%%%%%%%%%%%%%%%%%%%%%%%%%%%%%%%%%%%%%%%%%%%%%%%%%%%%%%%%%%%%%%%
\section{Application to Tucson, AZ}
\label{sec:water}

We now apply the formulation described in Sections \ref{sec:network_model} and \ref{sec:phi_divergences} to generate a generalized network \plp\ model describing water allocation in a developing region of Tucson, AZ shown in Figure \ref{fig:tucson_elevation}.

The majority of Tucson's water comes from the Colorado river, brought in by the (CAP) canal, which is indicated by the canal shown in blue in Figure \ref{fig:tucson_elevation}.
This water may then be treated and sent to customers, or pumped underground to one of five aquifers to be saved for future use.
One aquifer recharge facility, the Central Avra Valley Storage And Recovery Project (CAVSARP) is shown in Figure \ref{fig:tucson_elevation}, along with two water treatment plants: the Tucson Water Reclamation Facility (TWRF) and the Haden-Udal Water Treatment Plant (HUWTP).

One southeastern portion of Tucson, outlined in purple in Figure \ref{fig:tucson_elevation}, is being increasingly developed.
This area is split into different elevation zones, shown by the green lines, the demands in each of which will be served by a dedicated reservoir and pump.
That is, the elevation lines in Figure \ref{fig:tucson_elevation} also split the region into several pressure zones.
Given the capacity of the existing treatment plants, and the energy cost of pumping the water through a series of pressure zones, Tucson Water is interested in the possibility of building additional treatment facilities  in this area, herein known as the RESIN (REsilient and Sustainable INfrastructures) area.

\begin{figure}
	\FIGURE
	{%
		\includegraphics*[width=.6\textwidth]{tucson_water_images/tucson_elevation.png}%
	}
	{
		A map of Tucson, AZ, showing the RESIN area outlined in purple in the southeastern part.
		The green lines indicate elevation changes, denoting the difference between the pressure zones considered in this study.
		\label{fig:tucson_elevation}
	}
	{}
\end{figure}

A more detailed view of Tucson's water treatment system is shown in Figure \ref{fig:tucson_treatment}.
The blue nodes and arcs indicate the potable water treatment, storage and distribution system, which takes CAP water as an input and distributes it to the pressure zones shown in Figure \ref{fig:tucson_zones}.
The purple nodes and arcs indicate non-potable water which can be used to satisfy some (especially agricultural and industrial) demands.
The black nodes and arcs indicate waste water that is returned from potable water use.

\begin{figure}
	\FIGURE
	{%
		\includegraphics*[width=.8\textwidth]{tucson_water_images/nodes_central.png}%
	}
	{
		A schematic of the existing water treatment and recharge facilities in Tucson.
		Blue indicated potable water, purple indicates non-potable (reclaimed) water, and black represents waste water.
		\label{fig:tucson_treatment}
	}
	{}
\end{figure}

The pressure zones in the RESIN area, shown in Figure \ref{fig:tucson_zones}, have a largely period structure.
Each pressure zone contains a potable and non-potable demand node.
Note that potable water, being of higher quality, can be used to meet either type of demand.
Each zone also contains a potable-water reservoir and pump, which is used to supply potable to the next pressure zone, and a booster station that provides the conveyance for the non-potable water.
Finally, each zone has a wastewater return, which is transported back to the centralized processing facilities via a gravity flow.

The quantity of water demand in each zone is determined as a multiple of the projected population in each year.
Potable demand makes up 80\% of the total demand, and nonpotable demand is the remaining 20\%.

\begin{figure}
	\FIGURE
	{%
		\begin{tabular}{c}
			\includegraphics*[width=.5\textwidth]{tucson_water_images/zones_c_e.png}%
			\includegraphics*[width=.5\textwidth]{tucson_water_images/zones_split.png} \\
			\includegraphics*[width=.6\textwidth]{tucson_water_images/zones_legend.png}
		\end{tabular}
	}
	{
		A schematic illustration of the pressure zones in the RESIN area.
		Each zones contains potable and non-potable demand nodes, a reservoir and booster station for transporting the potable and non-potable water, respectively, and waste water return pipes.
		\label{fig:tucson_zones}
	}
	{}
\end{figure}

Finally, additional infrastructure may be constructed in the RESIN area.
One option is to construct a satellite wastewater treatment plant (WWTP), capable of treating wastewater up to a nonpotable quality, for satisfying demands in it's own zone and and higher zones.
Additionally, an indirect potable reuse (IPR) facility can be constructed, which further treats water from the WWTP up to potable quality.
This IPR facility can include reverse osmosis (RO) or not.
Figure \ref{fig:tucson_zones_wwtp} gives an illustration of an additional WWTP and IPR constructed in zone GS.

\begin{figure}
	\FIGURE
	{%
		\begin{tabular}{c}
			\includegraphics*[width=.8\textwidth]{tucson_water_images/zones_south_ipr.png}%
		\end{tabular}
	}
	{
		A satellite wastewater treatment plant is shown here, build in pressure zone GS.
		\label{fig:tucson_zones_wwtp}
	}
	{}
\end{figure}

The model has a total of $P = 41$ time periods, representing years 2010--2050.
For each time period, the network has 62 nodes representing demand for potable and nonpotable (reclaimed) water, pumps, water treatment plants, and the available water supply from the Colorado River.
The network in each time period has 102 arcs, representing the pipe network carrying the water between the nodes physically and connecting the network to the five reservoirs that connect the time stages in the model.
We use $P_1 = 5$ time periods for the first stage.

Uncertainty in the model comes from population growth, CAP water availability, and climate-driven per-capita demand.
The remainder of this section discusses our methods for modeling the problem uncertainty.

\subsection{Population \& Water Supply Estimates}

\begin{table}
	\TABLE
	{
		Population estimates for the RESIN area, and for the Tucson Water service area.
		\label{tb:population}
	}
	{\begin{tabular}{lcc}
		\hline
		         & \multicolumn{2}{c}{population} \\
		Estimate & 2010 & 2050 \\
		\hline
		\cite{taz}    &  53,028 &   460,000 \\
		\cite{wisp}   &  53,028 &   740,000 \\
		\hline
		\cite{tucson} & 750,000 & 1,300,000 \\
		\hline
	\end{tabular}}
	{}
\end{table}

Estimates of the population in each pressure zone were compiled from two population studies in Pima County in Arizona: the Water \& Wastewater Infrastructure, Supply \& Planning Study \citep{wisp} and the Traffic Analysis Zone (TAZ) \citep{taz}.
\cite{taz} provides a lower estimate of growth in the RESIN area, while \cite{wisp} provides the higher estimate.
The RESIN area represents a subset of the population served by Tucson Water, the population of which is estimated from \cite{tucson}.
Sample values can be seen in Table \ref{tb:population}.

Tucson Water has a yearly CAP water allocation of 144,000 acre-feet, which may be reduced in the event of an extended drought reducing the flows of the Colorado river.
The Arizona Department of Water Resources believes that a 10\% reduction to 130,000 acre-feet is the maximum realistic reduction by 2050 \citep{scott2012scenario}.
The water allocation to the RESIN area is determined by the proportion of the Tucson Water service population in the RESIN area, e.g., under the \cite{wisp} growth scenario with the full CAP allocation, the RESIN area is allocated about 81,970 acre-feet in 2050.

\subsection{Water Demand Regression}
\label{ssec:demand_regression}

We next need to estimate the weather-dependent demand of water resources in the RESIN area, which will be integrated climate models to project the future demand for water in the developing area.
To estimate demand separately in each elevation zone, we need a statistical model of the per-capita water demand to be used in conjunction with the zonal population estimates.
For this task, data on monthly demand for water in Tucson from 1991-2011 was obtained from Tucson, along with the recorded amount of rain and the average daily high temperature in each month.

A direct measure of population was only available on a yearly basis, but service counts were available for each month, from which population estimates were inferred in order to estimate the per-capita demand for water.
We performed a linear regression on the total population against the monthly service count data to estimate the population for each month.
The linear regression found that only the number of single family residences was a statistically significant predictor of population size.
The results are shown in Table \ref{tb:population_linear_reg}.
The counts of single family residences accounted for the vast majority of the variation in the population data, demonstrated by an adjusted $R^2 = 0.9946$.

\begin{table}
	\TABLE
	{
		Results of the linear regression of population on number of single family units in Tucson Water's service counts.
		\label{tb:population_linear_reg}
	}
	{\begin{tabular}{rrrrr}
		\hline
						   & Estimate  & Std. Error & $t$ value & $p$ \\
		\hline
		(Intercept)    & 2.026e+05 & 7.145e+03  & 28.3 6    & $<2 \cdot 10^{-16}$ \\
		Single Family  & 2.543e+00 & 4.171e-02  & 60.98     & $<2 \cdot 10^{-16}$ \\
		\hline
	\end{tabular}}
	{}
\end{table}

Given the monthly estimates on population, the water demand per capita in gallons per capita per day (GPCD) were calculated for every month in the 1991-2011 data.
A linear regression of these GPCD values on the mean daily high temperature and average precipitation rate for each month, and the year of the observation, was then calculated.
The year of the observation is included because the average GPCD water demand began dropping near the beginning of the 21st century, from over 170 GPCD in 1996 to under 140 GPCD in 2011.
The linear fit, shown in Table \ref{tb:gpcd_linear_reg} has adjusted $R^2 = 0.7266$.

\begin{table}
	\TABLE
	{
		Results of the linear regression of water demand in GPCD on average daily high temperature, precipitation rate, and year.
		\label{tb:gpcd_linear_reg}
	}
	{\begin{tabular}{rrrrr}
		\hline
						   & Estimate  & Std. Error & $t$ value & $p$ \\
		\hline
		(Intercept)    & 2721.2539 & 355.5267   & 7.654     & $4.33\cdot 10^{-13}$ \\
		Temperature    &   3.5830  & 0.1446     & 24.774    & $< 2\cdot 10^{-16}$\\
		Precipitation  &  -2.6329  & 1.1337     & -2.322    & $0.021$ \\
		Year           &  -1.3296  & 0.1776     & -7.486    & $1.24\cdot 10^{-12}$ \\
		\hline
	\end{tabular}}
	{}
\end{table}

This fit, with the year included, projects water demand to be around 145-155 GPCD in 2010, which decreases over time to 100-110 GPCD by 2050.
Tucson Water uses a typical estimate of 120-145 GPCD in their projections.
The range specified by this fit is similar to this for many of the years considered.
Because the water demand is linear in the year, we refer to this as the linear fit.

The linear fit for projecting a continual increase in water efficiency out to 2050.
To combat this, a second model was developed to capture the decrease in water usage in the data, but not extrapolate this trend.
For this, we choose to conduct the regression on a bounded value of the year, i.e., choose the value in a set $\{Y_l, Y_l+1, \dots, Y_u\}$ closest to the actual year.
First, the upper bound $Y_u = 2011$ was chosen so that no decrease in water demand would be projected beyond the scope of the data.
The value $Y_l = 2004$ was then selected to generate the best fit to the data.
This piecewise fit is shown in Table \ref{tb:gpcd_piecewise_reg}, and has $R^2 = 0.7634$.

\begin{table}
	\TABLE
	{
		Results of the linear regression of water demand in GPCD on average daily high temperature, precipitation rate, and and the bounded year.
		\label{tb:gpcd_piecewise_reg}
	}
	{\begin{tabular}{rrrrr}
		\hline
						   & Estimate  & Std. Error & $t$ value & $p$ \\
		\hline
		(Intercept)    & 9254.0201 & 904.3062   & 10.233    & $<2\cdot 10^{-16}$ \\
		Temperature    & 3.5913    & 0.1345     & 26.694    & $<2\cdot 10^{-16}$ \\
		Precipitation  & -2.2570   & 1.0510     & -2.148    & $0.0327$ \\
		Bounded Year   & -4.5847   & 0.4509     & -10.167   & $<2\cdot 10^{-16}$ \\
		\hline
	\end{tabular}}
	{}
\end{table}

The piecewise fit typically predicts demands of 135-145 GPCD throughout 2010-2050, increasing through the time period, which is in-line with Tucson Water's predictions.

\subsection{Climate Models Used}

A list of the climate models used in this paper is shown in Table \ref{tb:climate_models}.
Bias-Corrected and Spatially Downscaled (BCSD) data from Coupled Model Intercomparison Project: Phase 5 (CMIP5) was obtained from \citep{cmip5}.
Additional information on the downscaled data can be found in \citep{bcsdreclamation}.
Each model included predictions of the variables \texttt{tasmax} and \texttt{pr}, the average daily high temperature ($^\circ C$) and the average precipitation rate (mm/day) in each month.

\begin{table}
	\TABLE
	{
		A list of climate models used in this analysis
		\label{tb:climate_models}
	}
	{\begin{tabular}{p{.7\textwidth}|r}
		\hline
		Institution & Model \\
		\hline
		\hline
		Commonwealth Scientific and Industrial Research Organization (CSIRO) and Bureau of Meteorology (BOM), Australia & CSIRO-mk-3-6-0 \\
		\hline
		\multirow{2}{*}{Geophysical Fluid Dynamics Laboratory} & GFDL-CM3 \\
		 & GFDL-ESM2M \\
		\hline
		Met Office Hadley Centre & HadGEM2-ES \\
		\hline
		\multirow{3}{*}{\vbox{Atmosphere and Ocean Research Institute (The University of Tokyo), National Institute for Environmental Studies, and Japan Agency for Marine-Earth Science and Technology}}  & MIROC5 \\
		 & MIROC-ESM \\
		 & \\
		\hline
	\end{tabular}}
	{}
\end{table}

Each model had output associated with a given path for future greenhouse gas emissions.
Our analysis includes four paths: RCP2.6, RCP4.5, RCP6.0 and RCP8.5.
RCP2.6 is an optimistic scenario where emissions are drastically reduced by mid-century.
RCP4.5 and RCP6.0 show stabilization of emissions before and after 2100, respectively.
Finally, RCP8.5 is the scenario where emissions continue to grow quickly throughout the remainder of the century.
A comparison of the model output of the average daily high temperature for 2040-2050 for emissions path RCP8.5 is shown in Table \ref{tb:model_comparison_temperature}.

\begin{table}
	\TABLE
	{
		A comparison of projected ($\geq 2040$) average daily high temperatures ($^\circ C$) with the historical record (1991-2011) in the RESIN area.
		All values come from models using emissions path RCP8.5.
		\label{tb:model_comparison_temperature}
	}
	{\csvautotabular{tables/temperature.csv}}
	{}
\end{table}




%%%%%%%%%%%%%%%%%%%%%%%%%%%%%%%%%%%%%%%%%%%%%%%%%%%%%%%%%%%%%%%%%%%%%%%%%%%%%%%%
\section{Computational Results}
\label{sec:comp}

The water allocation model was solved using the decomposition algorithm described in \cite{love2013phi}.
We solved the model for different $\phi$-divergences, using values of $\rho$ equivalent to asymptotic confidence regions of $90\%$, $95\%$ and $99\%$.

Figure \ref{fig:shortage_frequency_piecewise} shows the total shortage amount over the 41-year study period for each infrastructure configuration when using the piecewise fit described in Table \ref{tb:gpcd_piecewise_reg}.
We can see that a satellite wastewater treatment plant provides the most substantial reduction in shortage severity, but that the indirect potable recharge facility decreases the worst-case shortages.
The reverse osmosis (RO) option offers little benefit over the generic IPR.
Figure \ref{fig:shortage_frequency_linear} shows the shortage frequency from the linear fit, for which we see a similar pattern, although with smaller shortages due to the lower water demand per capita through most of the study period.
These shortage amounts are computed with the Modified $\chi^2$ $\phi$-divergence, but are dictated largely by the physical constraints of the system, and do not change substantially with different $\phi$ or $\rho$.

\begin{figure}
	\FIGURE
	{%
		\begin{tabular}{c}
			\includegraphics*[width=.9\textwidth]{images/piecewise_shortage_frequency}%
		\end{tabular}
	}
	{
		A histogram of total shortage amount over the 41-year study period for each infrastructure configuration with the piecewise water demand fit described in Table \ref{tb:gpcd_piecewise_reg}.
		\label{fig:shortage_frequency_piecewise}
	}
	{}
\end{figure}

\begin{figure}
	\FIGURE
	{%
		\begin{tabular}{c}
			\includegraphics*[width=.9\textwidth]{images/linear_shortage_frequency}%
		\end{tabular}
	}
	{
		A histogram of total shortage amount over the 41-year study period for each infrastructure configuration with the linear water demand fit described in Table \ref{tb:gpcd_linear_reg}.
		\label{fig:shortage_frequency_linear}
	}
	{}
\end{figure}

To compare the models and emissions scenarios, we computed the total probability assigned to each.
Results for the Kullback-Leibler divergence can be seen in Table \ref{tb:pworst_model_emission_kl}.
We note that the highest probability is assigned to the highest emissions scenario, RCP8.5, but that the second highest probability is given to RCP4.5, a somewhat lower emissions scenario than RCP6.0.
RCP6.0 in fact has the lowest anticipated costs over the 2010-2050 time period.
We can also see that GFDL-CM3 results in the highest running cost overall for the water distribution system, followed by MIROC-ESM-CHEM.
MIROC5 is tends to generate lower-cost simulations, and thus is given the lowest probability, followed by CSIRO.

Considering both climate model and emissions scenario together, we find that GFDL-CM3 with the RCP4.5 results in the highest operating cost, while MIROC5 with RCP6.0 has the lowest cost.

\begin{table}
	\TABLE
	{
		Marginal worst-case probabilities of each climate model and emissions scenario as computed by the Kullback-Leibler Divergence.
		\label{tb:pworst_model_emission_kl}
	}
	{\begin{tabular}{l|cccc|c}
		 & RCP2.6 & RCP4.5 & RCP6.0 & RCP8.5 & (all) \\
\hline
CSIRO & 0.0381 & 0.0416 & 0.0354 & 0.0404 & 0.1554 \\
GFDL-CM3 & 0.0473 & 0.0522 & 0.0406 & 0.0428 & 0.1829 \\
GFDL-ESM2M & 0.0411 & 0.0406 & 0.0367 & 0.0427 & 0.1611 \\
HadGEM2-ES & 0.0407 & 0.041 & 0.0429 & 0.0436 & 0.1683 \\
MIROC5 & 0.0353 & 0.0402 & 0.0344 & 0.0426 & 0.1524 \\
MIROC-ESM-CHEM & 0.0395 & 0.0432 & 0.0466 & 0.0505 & 0.1799 \\
\hline
(all) & 0.2421 & 0.2587 & 0.2366 & 0.2626 & 1 \\

	 \end{tabular}}
	{}
\end{table}

Table \ref{tb:shortage_model_emission_kl} lists the maximum total shortage over the 2010-2050 projection for each climate projection.

\begin{table}
	\TABLE
	{
		Worst-case shortage for each climate model and emissions scenario, with the IPR.
		\label{tb:shortage_model_emission_kl}
	}
	{\begin{tabular}{l|cccc|c}
		 & RCP2.6 & RCP4.5 & RCP6.0 & RCP8.5 \\
\hline
CSIRO mk3.6 & 426600 & 454260 & 416560 & 415610 \\
GFDL-CM3 & 404340 & 458710 & 408000 & 443190 \\
GFDL-ESM2M & 414110 & 416580 & 422730 & 411840 \\
HadGEM2-ES & 440650 & 422440 & 436320 & 446510 \\
MIROC5 & 426320 & 449710 & 435780 & 403610 \\
MIROC-ESM-CHEM & 398940 & 433990 & 415510 & 451110 \\

	 \end{tabular}}
	{}
\end{table}

Table \ref{tb:total_cost} calculates the worst-case expected operating cost of the water system from 2010-2050.
This cost includes \$800 loss per acre-foot of shortage.
We can see that, while the satellite wastewater treatment plant decreases the cost only by \$10-\$15 million, the IPR facility reduces the cost by an additional \$25 million over the 41-year time span.

\begin{table}
	\TABLE
	{
		Total worst-case expected operating cost of the water system over the 2010-2050 time span.
		Columns represent worst-case costs at the 90\%, 95\% and 99\% confidence levels.
		\label{tb:total_cost}
	}
	{\begin{tabular}{l|cccc}
		Infrastructure & \multicolumn{3}{c}{Total Operating Cost (\$M)} \\
		     & 90\%   & 95\%   & 99\% \\
		\hline
		None & 319.34 & 320.06 & 321.41 \\
		WWTP & 303.91 & 304.59 & 305.85 \\
		IPR  & 279.86 & 280.55 & 281.82 \\
		RO   & 290.89 & 291.60 & 292.90
	 \end{tabular}}
	{}
\end{table}

Finally, we applied the value of data condition developed in Section \ref{ssec:value} to the results of the water model.
Several themes were immediately clear from the data: (1) every low population scenario satisfied the condition, and (2) high population, low supply scenarios only rarely satisfied the condition.
This has the reasonable interpretation that additional confidence in low population growth will reduce the expected cost of operating the water distribution system by reducing the amount of water that must be processed.
Likewise, the high population, low scenario is more expensive to run, and generates more penalties from shortage.
Additional confidence in these scenarios might decrease expected cost only if they are associated with more optimistic temperature and precipitation predictions.

The remaining case, with high population and high supply, is summarized in Table \ref{tb:value}.
We note that the value of data condition is more likely to be satisfied by the lower emissions scenarios, with the exception of the GFDL-CM3 model.
Again, this is a fairly intuitive result, as higher emissions scenarios will eventually lead to higher temperatures and increased water demand.
%We further note that all the scenarios that satisfy the condition with the Burg entropy also satisfy the condition with the Modified $\chi^2$ distance.

\begin{table}
	\TABLE
	{
		An indication of which climate models satisfy the value of data condition from Section \ref{ssec:value}.
		Results are presented for Burg entropy (b) and Modified $\chi^2$ distance (m).
		\label{tb:value}
	}
	{\begin{tabular}{l|cccc}
		               & RCP2.6    & RCP4.5    & RCP6.0    & RCP8.5 \\
		\hline
		CSIRO mk3.6    & (b,m)     & (b,m)     & (b,m)     & (b,m)     \\
		GFDL-CM3       &           &           & (b,m)     & (m)       \\
		GFDL-ESM2M     & (b,m)     & (b,m)     & (b,m)     & (m)       \\
		HadGEM2-ES     & (b,m)     & (b,m)     & (m)       & (m)       \\
		MIROC5         & (b,m)     & (b,m)     & (b,m)     & (m)       \\
		MIROC-ESM-CHEM & (b,m)     & (m)       &           &           \\
	 \end{tabular}}
	{}
\end{table}

%%%%%%%%%%%%%%%%%%%%%%%%%%%%%%%%%%%%%%%%%%%%%%%%%%%%%%%%%%%%%%%%%%%%%%%%%%%%%%%%
\section{Summary and Future Work}
\label{sec:concl}

In this paper, we have integrated projections for future climate, population growth and available water supply into a single integrated water allocation model.
These scenarios were combined in a \plp\ model proposed in \cite{love2013phi}, for several $\phi$-divergences.
We applied this method to a developing area in Tucson, Arizona, and determined projections for the severity of future water shortages, the expected operating cost of the system, and a worst-case distribution on the future scenarios.

Future extensions of this study have several possible paths.
First, we would like to extend this method to the multistage setting, where uncertainty is revealed in multiple stages, rather than all occurring simultaneously.
This will allow for an extended treatment of projections of population and water availability.
Integrating the climate projections into the estimates of CAP water availability would help to represent the correlation between demand and supply better.
Finally, increasing the size of the model to include more of Tucson and the state of Arizona would allow for more detailed future water planning.


% Acknowledgments here
% \ACKNOWLEDGMENT{%
% This work has been partially supported by the National Science Foundation through grant CMMI-1345626. We also gratefully acknowledge support provided by a Water Sustainability Program Fellowship through the Technology and Research Initiative Fund at the University of Arizona.
% }% Leave this (end of acknowledgment)


% Appendix here
% Options are (1) APPENDIX (with or without general title) or 
%             (2) APPENDICES (if it has more than one unrelated sections)
% Outcomment the appropriate case if necessary
%

\section*{Acknowledgements}
Support provided by a Water Sustainability Program Fellowship through the Technology and Research Initiative Fund at the University of Arizona.
This work has also been partially supported by the National Science Foundation through grant CMMI-1151226.

We acknowledge the World Climate Research Programme's Working Group on Coupled Modelling, which is responsible for CMIP, and we thank the climate modeling groups (listed in Table \ref{tb:climate_models} of this paper) for producing and making available their model output.
For CMIP the U.S. Department of Energy's Program for Climate Model Diagnosis and Intercomparison provides coordinating support and led development of software infrastructure in partnership with the Global Organization for Earth System Science Portals.

% References here (outcomment the appropriate case) 

% CASE 1: BiBTeX used to constantly update the references 
%   (while the paper is being written).
%\bibliographystyle{ijocv081} % outcomment this and next line in Case 1
\bibliography{love_lro} % if more than one, comma separated

% CASE 2: BiBTeX used to generate mypaper.bbl (to be further fine tuned)
%\input{mypaper.bbl} % outcomment this line in Case 2

%\section*{Author Biographies}

%\noindent {\bf DAVID LOVE} is a graduate student in the Graduate Interdisciplinary Program in Applied Mathematics at the University of Arizona.
%His research interests include distributionally robust stochastic programming and water resources management.
%His email address is \url{dlove@math.arizona.edu} and his web page is \url{http://math.arizona.edu/~dlove}.

%\bigskip

%\noindent {\bf G\"{U}ZIN BAYRAKSAN} is an Associate Professor of Integrated Systems Engineering at the Ohio State University.
%Her research interests include Monte Carlo sampling methods for stochastic programming and applications to water resources.
%Her email address is \url{bayraksan.1@osu.edu} and her web page is \url{http://www-iwse.eng.ohio-state.edu/biosketch_GBayraksan.cfm}

\end{document}

