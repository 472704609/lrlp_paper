%%%%%%%%%%%%%%%%%%%%%%%%%%%%%%%%%%%%%%%%%%%%%%%%%%%%%%%%%%%%%%%%%%%%%%%%%%%%
%% Author template for INFORMS Journal on Computing (ijoc)
%% Mirko Janc, Ph.D., INFORMS, mirko.janc@informs.org
%% ver. 0.95, December 2010
%%%%%%%%%%%%%%%%%%%%%%%%%%%%%%%%%%%%%%%%%%%%%%%%%%%%%%%%%%%%%%%%%%%%%%%%%%%%
%\documentclass[ijoc,blindrev]{informs3}
\documentclass[ijoc,letterpaper]{informs3} % current default for manuscript submission

%%\OneAndAHalfSpacedXI
%%\OneAndAHalfSpacedXII % current default line spacing
\DoubleSpacedXII
%%\DoubleSpacedXI

% If hyperref is used, dvi-to-ps driver of choice must be declared as
%   an additional option to the \documentclass. For example
%\documentclass[dvips,ijoc]{informs3}      % if dvips is used
%\documentclass[dvipsone,ijoc]{informs3}   % if dvipsone is used, etc.

% Private macros here (check that there is no clash with the style)

% Natbib setup for author-year style
\usepackage{natbib}
 \bibpunct[, ]{(}{)}{,}{a}{}{,}%
 \def\bibfont{\small}%
 \def\bibsep{\smallskipamount}%
 \def\bibhang{24pt}%
 \def\newblock{\ }%
 \def\BIBand{and}%

%% Setup of theorem styles. Outcomment only one. 
%% Preferred default is the first option.
\TheoremsNumberedThrough     % Preferred (Theorem 1, Lemma 1, Theorem 2)
%\TheoremsNumberedByChapter  % (Theorem 1.1, Lema 1.1, Theorem 1.2)

%% Setup of the equation numbering system. Outcomment only one.
%% Preferred default is the first option.
\EquationsNumberedThrough    % Default: (1), (2), ...
%\EquationsNumberedBySection % (1.1), (1.2), ...

% In the reviewing and copyediting stage enter the manuscript number.
%\MANUSCRIPTNO{} % When the article is logged in and DOI assigned to it,
                 %   this manuscript number is no longer necessary

\usepackage{amsmath}
\usepackage{amssymb}
\usepackage{setspace}
\usepackage{paralist}
\usepackage{graphicx}
\usepackage{url}
\usepackage{algorithm}
\usepackage{algpseudocode}

% Frequently used general mathematics
\newcommand{\R}{{\mathbb{R}}}
\newcommand{\Rp}{\R^+}
\newcommand{\Z}{{\mathbb{Z}}}
\newcommand{\Zp}{\Z^+}
\newcommand{\Q}{\mathbb{Q}}
\newcommand{\N}{\mathbb{N}}

% Commands for probability
\renewcommand{\P}{\mathbb{P}}
\newcommand{\E}{\mathbb{E}}
\newcommand{\p}[1]{\P \left[ #1 \right]}
\newcommand{\e}[1]{\E \left[ #1 \right]}
% \newcommand{\ee}[2]{\E_{#1} \left[ #2 \right]}

% Definitions of variables
\newcommand{\X}{X}
\newcommand{\x}{\mathbf{x}}
\newcommand{\xh}{\hat{\x}}
\newcommand{\lh}{\hat{\lambda}}
\newcommand{\mh}{\hat{\mu}}
\newcommand{\xs}{\x^*}
\newcommand{\xit}{\tilde{\mathbf{\xi}}}
\newcommand{\zt}{\tilde{z}}
\newcommand{\zs}{z^*}

% Further variables
\newcommand{\y}{\mathbf{y}}
\renewcommand{\c}{\mathbf{c}}
\renewcommand{\k}{\mathbf{k}}
\renewcommand{\b}{\mathbf{b}}
\renewcommand{\d}{\mathbf{d}}

% Epiconvergence for \plp
\newcommand{\ptrue}{p^{\text{true}}}

% Useful mathematics functions
\newcommand{\keywords}[1]{\par\noindent\enspace\ignorespaces\textbf{Keywords:} #1}
% \newcommand{\keywords}[1]{\par\addvspace\baselineskip\noindent\keywordname\enspace\ignorespaces #1}
% \DeclareMathOperator*{\argmin}{argmin}
% \theoremstyle{plain}
% \newtheorem{theorem}{Theorem}
% \newtheorem{lemma}[theorem]{Lemma}
% \newtheorem{proposition}[theorem]{Proposition}
% \newtheorem{corollary}[theorem]{Corollary}
% 
% \theoremstyle{definition}
% \newtheorem{definition}[theorem]{Definition}
% 
% \theoremstyle{remark}
% \newtheorem{remark}[theorem]{Remark}
\newtheorem{property}[theorem]{Property}

\newcommand{\st}{\mbox{s.t.}}

% Naming shortcuts
\newcommand{\plp}{$\phi$LP-2}

\bibliographystyle{ijocv081}

%%%%%%%%%%%%%%%%
\begin{document}
%%%%%%%%%%%%%%%%

% Outcomment only when entries are known. Otherwise leave as is and 
%   default values will be used.
%\setcounter{page}{1}
%\VOLUME{00}%
%\NO{0}%
%\MONTH{Xxxxx}% (month or a similar seasonal id)
%\YEAR{0000}% e.g., 2005
%\FIRSTPAGE{000}%
%\LASTPAGE{000}%
%\SHORTYEAR{00}% shortened year (two-digit)
%\ISSUE{0000} %
%\LONGFIRSTPAGE{0001} %
%\DOI{10.1287/xxxx.0000.0000}%

% Author's names for the running heads
% Sample depending on the number of authors;
% \RUNAUTHOR{Jones}
% \RUNAUTHOR{Jones and Wilson}
% \RUNAUTHOR{Jones, Miller, and Wilson}
% \RUNAUTHOR{Jones et al.} % for four or more authors
% Enter authors following the given pattern:
\RUNAUTHOR{Love and Bayraksan}

% Title or shortened title suitable for running heads. Sample:
% \RUNTITLE{Bundling Information Goods of Decreasing Value}
% Enter the (shortened) title:
\RUNTITLE{Properties of Phi-Divergence Constrained Ambiguous Stochastic Programs}

% Full title. Sample:
% \TITLE{Bundling Information Goods of Decreasing Value}
% Enter the full title:
\TITLE{Phi-Divergence Constrained Ambiguous Stochastic Programs}

% Block of authors and their affiliations starts here:
% NOTE: Authors with same affiliation, if the order of authors allows, 
%   should be entered in ONE field, separated by a comma. 
%   \EMAIL field can be repeated if more than one author
\ARTICLEAUTHORS{%
\AUTHOR{David Love}
\AFF{University of Arizona, \EMAIL{dlove@email.arizona.edu}, \URL{http://math.arizona.edu/~dlove/}}
\AUTHOR{G\"{u}zin~Bayraksan}
\AFF{The Ohio State University, \EMAIL{bayraksan.1@osu.edu}, \URL{http://www-iwse.eng.ohio-state.edu/biosketch\_GBayraksan.cfm}}
% Enter all authors
} % end of the block

\ABSTRACT{%
	We investigate the properties of the $\phi$-divergence based ambiguous stochastic programs recently proposed by Ben-Tal et.\ al.
	We present a classification of $\phi$-divergences to elucidate their use for models with different properties.
	We examine the value of collecting additional data and the convergence to the associated non-ambiguous stochastic program.
	A decomposition-based solution algorithm to solve the resulting model is given.
	We apply the method to a water distribution problem in Tucson, Arizona.
}%

% Sample 
%\KEYWORDS{deterministic inventory theory; infinite linear programming duality; 
%  existence of optimal policies; semi-Markov decision process; cyclic schedule}

% Fill in data. If unknown, outcomment the field
\KEYWORDS{Ambiguous stochastic programming, distributionally robust optimization, phi-divergences, data-driven optimization}
%Optimization under uncertainty, water resources management,  ambiguous stochastic programming, robust optimization, environmental sustainability}
%\HISTORY{}

\maketitle
%%%%%%%%%%%%%%%%%%%%%%%%%%%%%%%%%%%%%%%%%%%%%%%%%%%%%%%%%%%%%%%%%%%%%%

% Samples of sectioning (and labeling) in IJOC
% NOTE: (1) \section and \subsection do NOT end with a period
%       (2) \subsubsection and lower need end punctuation
%       (3) capitalization is as shown (title style).
%
%\section{Introduction.}\label{intro} %%1.
%\subsection{Duality and the Classical EOQ Problem.}\label{class-EOQ} %% 1.1.
%\subsection{Outline.}\label{outline1} %% 1.2.
%\subsubsection{Cyclic Schedules for the General Deterministic SMDP.}
%  \label{cyclic-schedules} %% 1.2.1
%\section{Problem Description.}\label{problemdescription} %% 2.

\section{Introduction and Motivation}

In practice, many optimization problems can be modeled by stochastic programs minimizing an expected value of an uncertain objective function.
However, if the distribution of the uncertain parameters used in the model is incorrect, the stochastic program can give highly suboptimal results.
Such problems have led to the development of distributionally robust optimization, a modeling technique that replaces the probability distribution by a set of distributions, and optimizes the expected cost relative to the worst distribution in the uncertainty set.
One approach to this has been recently proposed by \cite{bental2011robust} is to use a set of distributions that are sufficiently small $\phi$-divergence from a given ``nominal'' distribution.
Of particular interest is the case when the nominal distribution is determined by observation by making it an empirical distribution.
In this paper, we adapt the $\phi$-divergence method to the setting of a two-stage stochastic linear program with recourse and call this the two-stage $\phi$-divergence linear program (\plp). 
We examine the properties of the resulting model and develop a simple condition for assessing the value of collecting extra data.
Finally, we present a modified Bender's Decomposition to solve the \plp and apply the above results to a water distribution planning problem.

\plp is an ambiguous stochastic program that is modeled on a two-stage minimax problem.
Stochastic programs with uncertain objective functions have long been studied by applying the minimax approach to an expected cost; see, e.g., \cite{zackova1966minimax,dupacova_87}.
\cite{shapiro2002minimax} and \cite{shapiro2004class} developed methods for converting stochastic minimax problems into equivalent stochastic programs with a certain distribution.

In recent years, there has been a growing interest in distributionally robust methods.
\cite{erdogan2006ambiguous} study chance-constrained stochastic programs where the set of distributions considered is determined by the Prohorov metric.
\cite{calafiore2005uncertain} develop a data-driven method for generating feasible solutions to chance constrained problems, and later \cite{calafiore2006distributionally} develop a method for converting distributionally robust chance constraints into second-order cone constraints.
\cite{pflug2007ambiguity} develop a data-driven method for solving a portfolio selection problem using the Kantorovich distance to define the set of distributions to consider.
\cite{jiang2012data} develop an exact approach to solving data-driven chance constrained programs.
\cite{delage2010distributionally} provide methods for modeling uncertain distributions of a specific form (e.g., Gaussian, exponential, etc.) or using moment-based constraints.

Three recent papers by \cite{wang2010likelihood}, \cite{calafiore2007ambiguous}, and \cite{hukullback} provide similar studies using a specific $\phi$-divergence that is described in Section \ref{sec:phi_divergences}.
All three make use of the Kullback-Leibler divergence to define the set of distributions.
Both \cite{wang2010likelihood} and \cite{hukullback} produce similar dual problems similar to that presented in \cite{bental2011robust} and used here.
\cite{hukullback} differs from this work and the others by considering a continuous distributions, but doesn't relate the nominal distribution to observational data.
Additionally, \cite{klabjan2013robust} uses the $\chi^2$ distance, another $\phi$-divergence, to define an uncertain demand distribution for a stochastic lot-sizing problem using historical data.
Our results also provide one quantification of the value of additional data, and apply the LRO method specifically to a two stage problem, and to water allocation.

The $\phi$-divergence method is an attractive data-driven approach because it uses the data directly; and only those data points or scenarios of interest are used in the calculations.
These scenarios can come from direct observation, results of simulation, from expert opinion regarding scenarios that the decision maker would especially like to be robust against.
Because the \plp depends only on these scenarios, the size of the problem is polynomial in the sample size, making it computationally tractable.

We apply \plp to a generalized network model of Colorado River water allocation in Tucson, Arizona motivated by the CALVIN (CALifornia Value Integrated Network) optimal water allocation model of California created by \cite{draper_etal_03}.
Other models of Colorado River water distribution have also been studied, such as the Colorado River Reservoir Model \cite{christensen2004effects} and the Colorado River Budget Model \cite{barnett2009sustainable}.
Our model is modified to incorporate ambiguous future uncertainty by using the $\phi$-divergence approach.

The contributions of our work are that it provides
\begin{inparaenum}[\itshape (i\upshape)]
	\item a classification of $\phi$-divergences and insight on which class is useful for certain model types,
	\item a simple condition to determine if an additional observation will change the worst-case distribution used in the optimal solution, 
	\item asymptotic analysis to discuss conditions under which the optimal value and solution set \plp will converge to the two-stage stochastic program with recourse under the true distribution, 
	\item a specialized decomposition-based algorithm to solve the resulting model,  and 
	\item application to water allocation problem under ambiguous uncertainty. 
\end{inparaenum}

This paper is organized as follows.
Section \ref{sec:phi_divergences} introduces the $\phi$-divergence and presents some useful properties.
Section \ref{sec:plp2} presents the derivation of $\phi$-divergence model for a two-stage stochastic program with recourse.
Sections \ref{sec:properties} describes properties of \plp, including its asymptotic properties, how to select the level of robustness, and an analysis of the value of collecting additional data;
Sections \ref{sec:classification} and \ref{sec:special_phi} present a classification of $\phi$-divergences and some special cases of $\phi$; 
In Section \ref{sec:soln_algorithm} we present a decomposition method for solving the \plp model; and in Section \ref{sec:comp_results} we present a generalized network water model and computational results for its \plp model.
Finally, we end in Section \ref{sec:concl} with conclusions and future work.

%%%%%%%%%%%%%%%%%%%%%%%%%%%%%%%%%%%%%%%%%%%%%%%%%%%%%%%%%%%%%%%%%%%%%%%%%%%%%%%%
\section{Introduction to $\phi$-Divergences}
\label{sec:phi_divergences}

In this section we define the concept of a $\phi$-divergence, and describe some of the properties that will be used through the remainder of the paper.
\cite{pardo2005statistical} provides a good overview of much of the known properties of $\phi$-divergences.
Many results in this section can be also found in \citep{bental2011robust}.

$\phi$-divergences are used in statistics to measure the ``distance'' between two distributions. 
In the discrete case, $\phi$-divergences can be used generally to measure the distance between two non-negative vectors $p = (p_1, \dots, p_n)^T$ and $q = (q_1, \dots, q_n)^T$, and specifically when $p$ and $q$ are probability vectors; i.e., satisfying $\sum_{\omega=1}^n p_\omega = \sum_{\omega=1}^n q_\omega = 1$.
The $\phi$-divergence is defined by
\[
	I_\phi(p,q) = \sum_{\omega=1}^n q_\omega \phi\left(\frac{p_\omega}{q_\omega}\right),
\]
where $\phi(t)$, called the $\phi$-divergence function, is a convex function on $t \geq 0$ such that $\phi(1) = 0$, and with the additional interpretations that $0 \phi(a/0) = a \lim_{t \rightarrow \infty} \frac{\phi(t)}{t}$, and $0 \phi(0/0) = 0$.
If both $p$ and $q$ are probability vectors, as we assume throughout this paper, we can additionally assume without loss of generality that $\phi(t) \geq 0$.
The function $\phi(t)$ can be modified as $\psi(t) = \phi(t) - c(t-1)$ with an appropriately chosen constant $c$ such that $\psi(t) \geq 0$ for all $t$, and $I_\psi(p,q) = I_\phi(p,q)$ for all probability vectors $p,q$.
If $\phi(t)$ is differentiable at $t = 1$ this can be done by selecting $c = \phi'(1)$; see, e.g., the Likelihood divergence and Burg entropy in Table~\ref{tb:phi_definitions}.

$\phi$-divergences are not, in general, metrics.
For example, most $\phi$-divergences do not satisfy the triangle inequality and many are not symmetric in the sense that $I_\phi(p,q) \neq I_\phi(q,p)$.
One exception is the Variation distance, which is equivalent to the $L^1$-distance between the vectors.

A $\phi$-divergence has an adjoint, defined by
\begin{equation} \label{eq:adjoint}
	\tilde{\phi}(t) = t \phi\left(\frac{1}{t}\right),
\end{equation}
which satisfies all criteria for a $\phi$-divergence \citep{bental1991certainty}, and has the property that $I_{\tilde{\phi}}(p,q) = I_\phi(q,p)$.
Divergences that are symmetric with respect to the input vectors are known as self-adjoint.

As is shown in \citep{bental2011robust}, the problem formulation involves use of the conjugate $\phi^* : \R \rightarrow \R \cup \{\infty\}$, defined as
\begin{equation} \label{eq:conjugate}
	\phi^*(s) = \sup_{t \geq 0} \{st - \phi(t)\}.
\end{equation}
The conjugate $\phi^*$ is a nondecreasing function, and may be undefined above some upper bound $\bar{s}$.

Table \ref{tb:phi_definitions} lists some common examples of $\phi$-divergences, along with their adjoints and conjugates.
For all divergences, $\phi(t) = \infty$ for $t < 0$, and the value of the conjugate is listed only in its domain; i.e., $\{s : \phi^*(s) < \infty\}$.
Table \ref{tb:phi_definitions} also lists a divergence, labeled ``Likelihood,'' that is somewhat different from the others.
The Likelihood divergence is equivalent to the Burg entropy when comparing probability vectors, but does not satisfy the normalizing condition $\phi_l(t) \geq 0$.
This divergence is included because \cite{wang2010likelihood} use it to formulate a distributionally robust newsvendor problem so that the ambiguity set of distributions have a sufficiently high empirical likelihood. 
They refer to this as the Likelihood Robust Optimization. 
We also note that \cite{calafiore2007ambiguous}, \cite{hukullback}, and \cite{wang2010likelihood} all use a different naming convention than the one given here, referring to the Likelihood divergence or Burg entropy as the ``Kullback-Leibler divergence''---reversing the order of the arguments $p$ and $q$ relative to the notation presented here.
In this paper, we use $q$ as the nominal distribution. 

\begin{table}
	\TABLE
	{
		Definitions of some common $\phi$-divergences, along with their adjoints $\tilde{\phi}(t)$ and conjugates $\phi^*(s)$.
		\label{tb:phi_definitions}
	}
	{\begin{tabular}{lccccc}
		\hline \\
		Divergence                        & $\phi(t)$          & $\tilde{\phi}(t)$               & $\phi(t), t \geq 0$   & $I_\phi(p,q)$     & $\phi^*(s)$ \\
		\hline
		Kullback-Leibler                  & $\phi_{kl}$        & $\phi_b$                        & $t\log t - t + 1$     & $\sum p_\omega \log\left(\frac{p_\omega}{q_\omega}\right)$ & $e^s - 1$ \\
		Burg Entropy                      & $\phi_b$           & $\phi_{kl}$                     & $-\log t + t - 1$     & $\sum q_\omega \log\left(\frac{q_\omega}{p_\omega}\right)$ & $-\log(1-s),\ s < 1$  \\
		Likelihood                        & $\phi_l$           & N.A.                      & $-\log t$             & $\sum q_\omega \log\left(\frac{q_\omega}{p_\omega}\right)$ & $-\log(-s) - 1,\ s < 0$ \\
		$\chi^2$-Distance                 & $\phi_{\chi^2}$    & $\phi_{m\chi^2}$                & $\frac{1}{t} (t-1)^2$ & $\sum \frac{(p_\omega-q_\omega)^2}{p_\omega}$              & $2 - 2\sqrt{1-s}$  \\
		Modified $\chi^2$-Dist.           & $\phi_{m\chi^2}$   & $\phi_{\chi^2}$                 & $(t-1)^2$             & $\sum \frac{(p_\omega - q_\omega)^2}{q_\omega}$            & $\begin{cases} -1 & s < -2 \\ s + \frac{s^2}{4} & s \geq -2 \end{cases}$ \\
% 		$\chi$-div,  $\theta > 1$ & $\phi_\chi^\theta$ & $t^{1-\theta}\phi_\chi^\theta$ & $|t-1|^\theta$         & $\sum q_\omega |1-\frac{p_\omega}{q_\omega}|^\theta$       & $\begin{cases} -1 & s \leq -\theta \\ s + (\theta-1)\left(\frac{|s|}{\theta}\right)^\frac{\theta}{\theta-1}  & s \geq -\theta \end{cases}$ \\
		Variation Distance                & $\phi_v$           & $\phi_v$                        & $|t-1|$               & $\sum |p_\omega - q_\omega|$                               & $\begin{cases} -1 & s \leq -1 \\ s & -1 \leq s \leq 1 \end{cases}$ \\
		Hellinger                         & $\phi_h$           & $\phi_h$                        & $(\sqrt{t} - 1)^2$    & $\sum (\sqrt{p_\omega} - \sqrt{q_\omega})^2$               & $\frac{s}{1-s},\ s < 1$ \\
	\hline
	\end{tabular}}
	{}
\end{table}

%%%%%%%%%%%%%%%%%%%%%%%%%%%%%%%%%%%%%%%%%%%%%%%%%%%%%%%%%%%%%%%%%%%%%%%%%%%%%%%%
\section{\plp\ Formulation}
\label{sec:plp2}

We begin with a two-stage stochastic linear program with recourse (SLP-2).
Let $\x$ be a vector of of first stage decision variables with cost vector $\c$, constraint matrix $A$ and right hand side $\b$.
We assume a finite distribution given by $p_\omega$ with scenarios indexed by $\omega = 1, \dots, n$.
The SLP-2 is
\begin{align}
	\min_\x \ & \left\{ \c\x + \sum_{\omega=1}^n p_\omega h_\omega(\x) : A\x = b, \x \geq 0 \right\} \label{eq:slp_first_stage}% \\
% 	\st \ & A\x = \b \nonumber  \\
% 	&\ \ \ \x \geq 0 \nonumber
\end{align}
where
\begin{align}
	h_\omega(\x) = \min_{y^\omega} \ & \left\{ \k^\omega y^\omega : D^\omega y^\omega = B^\omega \x + \d^\omega, y^\omega \geq 0 \right\} \label{eq:slp_second_stage}
\end{align}
We assume relatively complete recourse; i.e., the second-stage problems $h_\omega(\x)$ are feasible for every feasible solution $\x$ of the first-stage problem; and that the second-stage problems $h_\omega(\x)$ are dual feasible for every feasible solution $\x$ of the first-stage problem..

The SLP-2 formulation assumes that the distribution $\{p_\omega\}_{\omega=1}^n$ is known.
However, in many applications, including our water planning, the distribution is itself unknown.
One technique to deal with this is to replace the known distribution with an {\it ambiguity set} of distributions; i.e., a set of distributions which is believed to contain the true distribution.
\cite{bental2011robust} proposed that the ambiguity set could be generated with $\phi$-divergences, including those distributions whose divergence from the nominal distribution $q$ is sufficiently small.
Throughout much of this paper, we assume that $q$ is generated from observations, where scenario $\omega$ has been observed $N_\omega$ times, with $N = \sum_{\omega=1}^n N_\omega$ total observations, leading to the empirical distribution $\hat{p}^N_\omega = N_\omega / N$.
In SLP-2, this would correspond to probability of scenario $\omega$ to be $\hat{p}_\omega$.
By replacing the specific distribution in SLP-2 with a set of distributions with sufficiently close to the nominal distribution in the $\phi$-divergence sense, we create a model that we refer to as two-stage $\phi$-divergence linear program with recourse (\plp).

To derive the \plp, we begin by writing SLP-2 given in \eqref{eq:slp_first_stage}--\eqref{eq:slp_second_stage} in extensive form
\[
	\begin{array}{rrrl}
		\min_{\x,y^\omega} \ & \c\x & + \sum_\omega p_\omega \k^\omega y^\omega \label{eq:slp2cost} \\
		\st \ & A\x & & = \b \nonumber \\
		& -B^\omega \x & + D^\omega y^\omega & = \d^\omega,\ \forall \omega \nonumber \\
		& \x & & \geq 0 \nonumber \\
		& & y^\omega & \geq 0,\ \forall \omega. \nonumber
	\end{array}
\]
The SLP-2 formulation is then augmented by the set of distributions with sufficiently low divergence from $q$.
To be robust against all these possible distributions, the distribution that results in the maximum expected cost is considered.
Then, the objective function is minimized with respect to this worst-case distribution selected from the ambiguity set of distributions.
The resulting minimax formulation of \plp\ is
\begin{align}
	\min_{\x,y^\omega} \max_p \ & \c\x + \sum_\omega p_\omega \k^\omega y^\omega \label{eq:plp_primal}\\
	\st \ & A\x = \b, \x \geq 0 \nonumber \\
	& -B^\omega \x + D^\omega y^\omega = \d^\omega,\ \forall \omega \nonumber \\
	& \sum_\omega q_\omega \phi\left(\frac{p_\omega}{q_\omega}\right) \leq \rho \label{eq:plp_primal_divergence} \\
	& \sum_\omega p_\omega = 1 \label{eq:plp_primal_probability} \\
	& \x \geq 0 \nonumber \\
	& y^\omega, p_\omega \geq 0,\ \forall \omega. \label{eq:nonneg}
\end{align}

Taking the dual of the inner maximization problem, with dual variables $\lambda$ and $\mu$, of constraints (\ref{eq:plp_primal_divergence}) and (\ref{eq:plp_primal_probability}), respectively, yields
\begin{align*}
	\min_{\lambda,\mu} \ & \mu + \rho \lambda + \lambda \sum_\omega q_\omega \phi^*\left(\frac{\k^\omega y^\omega - \mu}{\lambda}\right) \\
	\st \ & \lambda \geq 0 \\
	& \frac{\k^\omega y^\omega - \mu}{\lambda} \leq \lim_{t \rightarrow \infty} \frac{\phi(t)}{t}, \ \forall \omega,
\end{align*}
See Theorem 1 of \cite{bental2011robust} for a derivation of the dual problem.
Note in particular that the dual formulation is accurate even for $q_\omega = 0$ for some $\omega$.

Combining the two minimizations gives \plp\ in extensive form
\begin{align}
	\min_{\x,\lambda,\mu,y^\omega} \ & \c\x + \mu + \rho \lambda + \sum_\omega q_\omega \phi^*\left(\frac{\k^\omega y^\omega - \mu}{\lambda}\right) \nonumber \\
	\st \ & A\x = \b \nonumber \\
	& -B^\omega \x + D^\omega y^\omega = \d^\omega,\ \forall \omega \label{eq:plp_det_equiv} \\
	& \frac{\k^\omega y^\omega - \mu}{\lambda} \leq \lim_{t \rightarrow \infty} \frac{\phi(t)}{t}, \ \forall \omega \nonumber \\
	& \x,\lambda,y^\omega \geq 0, \ \forall \omega. \nonumber
\end{align}

Finally, we wish to return the \plp\ to two-stage formulation.
\begin{align}
	\min_{\x,\lambda,\mu} \ & \c\x + \mu + \rho \lambda + \sum_\omega q_\omega \phi^*\left(\frac{h_\omega(\x) - \mu}{\lambda}\right) \nonumber \\
	\st \ & A\x = \b \label{eq:plp_two_stage} \\
	& \frac{\k^\omega y^\omega - \mu}{\lambda} \leq \lim_{t \rightarrow \infty} \frac{\phi(t)}{t}, \ \forall \omega \label{eq:plp_feas_constraint}\\
	& \x,\lambda \geq 0, \nonumber
\end{align}
where
\begin{align}
	h_\omega(\x) = \k^\omega y^\omega \label{eq:plp_second_stage} \\
	\st \ & -B^\omega \x + D^\omega y^\omega = \d^\omega \nonumber \\
	& y^\omega \geq 0. \nonumber
\end{align}

%%%%%%%%%%%%%%%%%%%%%%%%%%%%%%%%%%%%%%%%%%%%%%%%%%%%%%%%%%%%%%%%%%%%%%%%%%%%%%%%
\section{Properties of \plp}
\label{sec:properties}

In this section, we list some basic properties of \plp.
Most of these have already been noted in \cite{bental2011robust}, with a special case of LRO noted in \cite{wang2010likelihood}, but we list them for completeness and because some of these properties help with our special solution method.

\subsection{Basic Properties}

As noted in \cite{wang2010likelihood}, the \plp\ problem can be viewed as minimizing a coherent risk measure.
A coherent risk measure (in the basic sense), as defined in \cite{rockafellar2007coherent}, is a functional ${\cal R}: L^2 \rightarrow (-\infty,\infty]$ defined on random variables such that
\begin{enumerate}
	\item ${\cal R}(C) = C$ for all constants $C$,
	\item ${\cal R}((1-\lambda)X + \lambda X') \leq (1-\lambda){\cal R}(X) + \lambda {\cal R}(X')$, i.e., $\cal R$ is convex,
	\item ${\cal R}(X) \leq {\cal R}(X')$ when $X \leq X'$, i.e., $\cal R$ is monotonic,
	\item ${\cal R}(\lambda X) = \lambda {\cal R}(X)$ for $\lambda > 0$, i.e., $\cal R$ is positively homogeneous.
\end{enumerate}

\begin{property}
	\plp\ is equivalent to minimizing a coherent risk measure.
\end{property}

\begin{proof}
	Rockafellar also shows that $\cal R$ is a coherent risk measure if and only if it can be written using a risk envelope \cite{rockafellar2007coherent}.
	We will show that \plp\ can be written in the form of a risk envelope in the primal form (\ref{eq:plp_primal}) with the change of variables $\tilde{p}_\omega = \frac{p_\omega}{1/n}$.
	Throughout the proof, all expectations are taken with respect to the uniform distribution.
	
	First, probability constraint (\ref{eq:plp_primal_probability}) can be written as $\e{\tilde{p}} = 1$, where $\tilde{p}$ is the random variable taking values $\tilde{p}_\omega$ with equal probability.
	Then the $\phi$-divergence constraint (\ref{eq:plp_primal_divergence}) becomes $\sum_{\omega=1}^n \frac{1}{n} \tilde{p}_\omega \frac{\phi\left(\tilde{p}_\omega/\tilde{q}_\omega\right)}{\tilde{p}_\omega/\tilde{q}_\omega} \leq \rho$, where $\tilde{q}_\omega = \frac{q_\omega}{1/n}$ and $\frac{\phi(a/0)}{a/0} = \lim_{t \rightarrow \infty} \frac{\phi(t)}{t}$.
	Combining these yields the set ${\cal Q} = \{\tilde{p} | \e{\tilde{p}} = 1, \sum_{\omega=1}^n \frac{1}{n} \tilde{p}_\omega \frac{\phi\left(\tilde{p}_\omega/\tilde{q}_\omega\right)}{\tilde{p}_\omega/\tilde{q}_\omega} \leq \rho \}$, a closed and convex risk envelope.
	Finally, we can rewrite the inner maximization of (\ref{eq:plp_primal}) as $\max_{\tilde{p} \in {\cal Q}} \e{\tilde{p} h(x)}$, where $h(x)$ is the random variables defined by $\{h_\omega(x)\}$.
	Thus we see that \plp\ is the minimum of a coherent risk measure.
\end{proof}

\begin{remark}
	The above proof can be simplified by using $\tilde{p}_\omega = \frac{p_\omega}{q_\omega}$ if $q_\omega > 0$ for all $\omega$.
	However, the case of $q_\omega = 0$ plays an important role in the classification presented in Section \ref{sec:classification}.
\end{remark}


Note that being a coherent risk measure implies that \plp\ is a convex problem.
The convexity of LRO was also noted in \cite{wang2010likelihood}.

\begin{property}
	The second-stage problems are unaffected by the change from SLP-2 to \plp.
	This preservation of the time structure allows us to easily convert (sub-)derivatives of $h_\omega(\x)$ to (sub-)derivatives of $\phi^*\left(\frac{h_\omega(\x) - \mu}{\lambda}\right)$.
	We will use this in the decomposition method provided in Section \ref{sec:soln_algorithm}.
\end{property}

\begin{property}
	The appearance of the conjugate $\phi^*(s)$ in the objective of (\ref{eq:plp_two_stage}) gives the method retrieving the worst-case distribution,
	\begin{equation} \label{eq:p_worst}
		\frac{p_\omega}{q_\omega} \in \partial \phi^*\left(\frac{h_\omega(\x)-\mu}{\lambda}\right).
	\end{equation}
	If $q_\omega = 0$, then either $p_\omega = 0$ or $\infty \in \partial \phi^*\left(\frac{h_\omega(\x)-\mu}{\lambda}\right)$.
\end{property}

%%%%%%%%%%%%%%%%%%%%%%%%%%%%%%%%%%%%%%%%%%%%%%%%%%%%%%%%%%%%%%%%%%%%%%%%%%%%%%%%
\subsection{The Level of Robustness}
\label{ssec:robust_level}

The literature on $\phi$-divergences provides some insight on choosing a reasonable asymptotic value of $\rho$.
Theorem 3.1 of \cite{pardo2005statistical} shows that the statistic $T^\phi_N(\hat{p},\ptrue) = \frac{2N}{\phi''(1)} \sum_{\omega=1}^n \ptrue_\omega \phi\left(\frac{\hat{p}_\omega}{\ptrue_\omega}\right)$ converges in distribution to a $\chi^2$-distribution with $n-1$ degrees of freedom.
\cite{bental2011robust} then uses this result to suggest the asymptotic value
\begin{equation} \label{eq:asymptotic_rho}
	\rho = \frac{\phi''(1)}{2N} \chi^2_{n-1,1-\alpha},
\end{equation}
where $\chi^2_{n-1,1-\alpha}$ is the $1-\alpha$ percentile of a $\chi^2_{n-1}$ distribution, which produces an approximate $1-\alpha$ confidence region on the true distribution.

\subsection{The Value of Data} \label{ssec:value}

With a data driven formulation such as \plp, it is natural to ask how the behavior changes as more data is gathered.
In particular, for robust formulations like \plp\ one might be concerned about being overly conservative in the problem formulation and thus missing the opportunity to find a better solution to the true distribution.
For \plp, this means that the initial model is likely to be more conservative in an effort to be robust, while the new information could make the model less conservative because new information removes the current worst case distribution from the ambiguity set.  
In this section, we present a simple method of estimating the probability that taking an additional sample will eliminate the old worst-case distribution and allow for better optimization; i.e., a lower-cost solution.

To get such a condition, we must consider how $\rho$ changes as additional samples are taken, and use $\rho_N$ in this section to emphasize the dependence on sample size.
To be consistent with the known $\phi$-divergence results stated in Section \ref{ssec:robust_level}, we assume $\rho_N = \frac{\rho_0}{N}$.

\begin{proposition}
	An additional sample of scenario $\hat{\omega}$ will result in a decrease in the worst-case expected cost of the \plp\ if the following condition is satisfied
	\begin{equation} \label{eq:cost_decrease_cond}
		\sum_{\omega=1}^n q_\omega \phi^{*\prime}\left(\frac{N}{N+1}s^*_\omega\right) \left(\frac{N}{N+1}s^*_\omega\right) > \phi^*\left(\frac{N}{N+1}s^*_{\hat{\omega}}\right),
	\end{equation}
	where $s^*_\omega = \dfrac{h_\omega(\x^*_N) - \mu^*_N}{\lambda^*_N}$ and $(\x^*_N,\mu^*_N,\lambda^*_N)$ solves the $N$-sample problem.
\end{proposition}

\begin{proof}
	We begin this proof with the change of variables $\kappa = \frac{\lambda}{N}$, and note that $N\rho_N = \rho_0$ is constant.
	
	With this change of variables, the objective function is given by
	\[
		f_N(\x,\mu,\kappa) = c\x + \mu + \rho_0 \kappa + \sum_{\omega = 1}^n N_\omega \left[ \kappa \phi^*\left(\frac{h_\omega(\x) - \mu}{N\kappa} \right) \right].
	\]
	Let $z_N = \min_{\x,\mu,\kappa} f_N(\x,\mu,\kappa)$.
	We wish to find a simple estimate of the decrease in the optimal cost, $z_N - z_{N+1}$, associated with taking an additional sample of, say, $\hat{\omega}$, looking in particular for a condition under which $z_N - z_{N+1} > 0$.
	
	Let $(\x^*_N,\mu^*_N,\kappa^*_N)$ minimize $f_N$.
	Then $z_N - f_{N+1}(\x^*_N,\mu^*_N,\kappa^*_N)$ is a lower bound on the decrease in optimal cost $z_N - z_{N+1}$.
	We will find scenarios $\hat{\omega}$ such that $z_N - f_{N+1}(\x^*_N,\mu^*_N,\kappa^*_N) > 0$.

	The $N+1$ sample problem is
	\[
		cx + \mu + \rho_0 \kappa + \sum_{\omega = 1}^n N'_\omega \left[ \kappa \phi^*\left(\frac{h_\omega(x) - \mu}{(N+1)\kappa} \right) \right],
	\]
	where $N'_\omega$ is the number of observations of $\omega$ after $N+1$ total observations.
	Then the difference between the two optimal costs is
	\[
		\kappa \sum_{\omega=1}^n \left[ N_\omega \phi^*\left(\frac{h_\omega(x) - \mu}{N\kappa} \right) - N'_\omega \phi^*\left(\frac{h_\omega(x) - \mu}{(N+1)\kappa} \right) \right],
	\]
	which must be positive to guarantee a drop in optimal cost.
	Let $\hat{\omega}$ be the scenario observed on the next observation, then we can rewrite the condition as
	\begin{equation} \label{eq:raw_cond}
		\kappa \sum_{\omega=1}^n N_\omega \left[ \phi^*\left(\frac{h_\omega(x) - \mu}{N\kappa} \right) - \phi^*\left(\frac{h_\omega(x) - \mu}{(N+1)\kappa} \right) \right] - \kappa \phi^*\left(\frac{h_{\hat{\omega}}(x) - \mu}{(N+1)\kappa}\right) > 0.
	\end{equation}

	Let $s^N_\omega = \frac{h^\dagger_\omega(x) - \mu}{N\kappa}$ and $s^{N+1}_\omega = \frac{h^\dagger_\omega(x) - \mu}{(N+1)\kappa}$.
	The difference $\phi^*(s^N_\omega) - \phi^*(s^{N+1}_\omega)$ will be approximated by the derivative.
	Note that $\phi^*(s)$ is nondecreasing because $\phi(t)$ is confined to the first quadrant.
	First, for $s^N_\omega > 0$
	\begin{align*}
		\phi^{*\prime}(s^N_\omega) - \phi^{*\prime}(s^{N+1}_\omega) & \geq \phi^{*\prime}(s^{N+1}_\omega) |\Delta s|.
	\end{align*}
	Then for $s^N_\omega < 0$
	\begin{align*}
		\phi^{*\prime}(s^{N+1}_\omega) - \phi^{*\prime}(s^N_\omega) & \leq \phi^{*\prime}(s^{N+1}_\omega) |\Delta s|,
	\end{align*}
	and thus
	\begin{align*}
		\phi^{*\prime}(s^N_\omega) - \phi^{*\prime}(s^{N+1}_\omega) & \geq -\phi^{*\prime}(s^{N+1}_\omega) |\Delta s|.
	\end{align*}
	Furthermore, 
	\[
		|\Delta s| = | s^{N+1}_\omega - s^N_\omega | = \frac{|s^{N+1}_\omega|}{N},
	\]
	and so both cases reduce to
	\[
		\phi^{*\prime}(s^N_\omega) - \phi^{*\prime}(s^{N+1}_\omega) \geq \frac{1}{N} \phi^{*\prime}(s^{N+1}_\omega) s^{N+1}_\omega.
	\]

	Now we can guarantee (\ref{eq:raw_cond}) is satisfied with the condition
	\[
		\kappa \sum_{\omega=1}^n \frac{N_\omega}{N} \phi^{*\prime}(s^{N+1}_\omega) s^{N+1}_\omega - \kappa \phi^*\left(\frac{h^\dagger_{\hat{\omega}}(x) - \mu}{(N+1)\kappa}\right) > 0,
	\]
	or, rearranging and dividing by $\kappa > 0$,
	\begin{equation} \label{eq:main_value_derivation}
		\sum_{\omega=1}^n \frac{N_\omega}{N} \phi^{*\prime}(s^{N+1}_\omega) s^{N+1}_\omega > \phi^*(s^{N+1}_\omega).
	\end{equation}
	Finally, return to the original variables with the substitution $s^\omega_{N+1} = \frac{N}{N+1} \frac{h_\omega(\x^*_N) - \mu^*_N}{\lambda^*_N}$
\end{proof}

We can interpret \eqref{eq:cost_decrease_cond} as follows. If an additional sample is taken from the unknown distribution and the resulting observed scenario $\hat{\omega}$ satisfies (\ref{eq:cost_decrease_cond}), then the $(N+1)$-sample problem will have a lower cost than the $N$-sample problem that was already solved.
This is equivalent to saying that an additional observation of $\hat{\omega}$ will rule out the computed worst-case distribution given by $\{p_\omega\}$ given in \eqref{eq:p_worst}.

Next, we would like a lower bound on the probability that the next sample will decrease the optimal cost.
Let $L = \left\{ \hat{\omega} : \sum_{\omega=1}^n q_\omega \phi^{*\prime}\left(\frac{N}{N+1}s^*_\omega\right) \left(\frac{N}{N+1}s^*_\omega\right) > \phi^*\left(\frac{N}{N+1}s^*_{\hat{\omega}}\right) \right\}$.
That is, $L$ gives the set of scenarios that, if sampled one more observation, would result in a decrease in the optimal cost in \plp.

\begin{proposition}
	An approximate lower bound on the probability of selecting a scenario in the set $L = \left\{ \hat{\omega} : \sum_{\omega=1}^n q_\omega \phi^{*\prime}\left(\frac{N}{N+1}s^*_\omega\right) \left(\frac{N}{N+1}s^*_\omega\right) > \phi^*\left(\frac{N}{N+1}s^*_{\hat{\omega}}\right) \right\}$ can be found by solving 
	\begin{equation} \label{eq:prob_cost_decrease}
		-\min_{\mu,\lambda \geq 0} \left\{ \mu + \rho_N \lambda + q_L \lambda \phi^*\left(\frac{\mu-1}{\lambda}\right) + (1-q_L) \lambda \phi^*\left(\frac{\mu}{\lambda}\right) \right\},
	\end{equation}
	where $q_L = \sum_{\omega \in L} q_\omega$ is the probability of $L$ in the nominal distribution.
\end{proposition}

\begin{proof}
	We can estimate a lower bound on the probability of sampling a scenario in $L$ by using the same likelihood ambiguity set that was used to formulate \plp\ given in (\ref{eq:plp_primal_divergence}) to solve the minimization problem
	\begin{align}
		\min_{\omega \in L} \ & \sum_{\omega \in L} r_\omega \nonumber \\
		\mbox{s.t.} & \sum_\omega q_\omega \phi\left(\frac{r_\omega}{q_\omega}\right) \leq \rho_N \label{eq:lb_probability} \\
		& \sum_\omega r_\omega = 1 \nonumber \\
		& q_\omega \geq 0, \ \forall \omega, \nonumber
	\end{align}
	where we have introduced the dummy variables $r_\omega$ to distinguish the minimization in (\ref{eq:lb_probability}) from the worst-case distribution $\{p_\omega\}$ calculated in (\ref{eq:p_worst}). 
	Solving (\ref{eq:lb_probability}) yields an estimated lower bound on the probability that an additional sample will result a likelihood ambiguity set that does not contain the current worst-case distribution $\{p_\omega\}$ using the current set of observations.
	Note that $\min_{\omega \in L} \sum_{\omega \in L} r_\omega \leq q_L$, because the maximum likelihood distribution is always within the likelihood ambiguity set.
% 	We will use $\frac{N_L}{N}$ as a benchmark in Figures~\ref{fig:prob_cost_decrease_nd_n}, \ref{fig:prob_cost_decrease_gp} and \ref{fig:water_prob_decrease}. 

	We solve (\ref{eq:lb_probability}) by taking its dual, which results in the two dimensional nonlinear program (\ref{eq:prob_cost_decrease}).
\end{proof}

\subsection{Asymptotic Analysis of \plp}
\label{ssec:epiconvergence}

We now wish to show that the optimal value and solution of \plp\ converges to the optimal value and solution of the corresponding SLP-2 with the (unknown) true distribution $\ptrue$.
This convergence requires that the sequence of nominal distributions $q^N$ converge to the true distribution $\ptrue$ in $L^\infty$, a situation that is satisfied if $q^N$ is the empirical distribution.
In the proof, we assume $q^N = \hat{p}^N$.
We begin by showing that the worst-case distribution converges weakly to the true distribution as $N \rightarrow \infty$.

Let the probability space $(\Xi,{\cal F},\P^\infty)$ be the space associated with taking infinitely many samples from the distribution $\ptrue$.
Let $\Xi' \subset \Xi$ be a measure 1 set such that $\Vert \hat{p}^N(\xi) - \ptrue \Vert_\infty \rightarrow 0$.

\begin{proposition} \label{prop:weak_conv}
	Let $p \neq \ptrue$.
	For all $\epsilon > 0$ and $\xi \in \Xi'$ there exists $N'$ such that $\forall N \geq N'$ $I_{\phi}(p,\hat{p}^N) \leq \frac{\rho_0}{N} \Rightarrow \max_\omega |p_\omega - \ptrue_\omega| \leq \epsilon$.
\end{proposition}

\begin{proof}
	Let $Z = \{\omega : \ptrue_\omega = 0\}$ be the set of impossible scenarios.
	For simplicity, we assume $\epsilon$ is chosen so that $\max_{\omega \notin Z} \ptrue_\omega > \frac{\epsilon}{2}$.
	
	First, we note that $\max_\omega |p_\omega - \ptrue_\omega| \leq \max_\omega |p_\omega - \hat{p}^N_\omega| + \max_\omega |\hat{p}^N_\omega - \ptrue_\omega|$.
	Let $N''$ be such that $\max_\omega |\hat{p}^N_\omega - \ptrue_\omega| \leq \frac{\epsilon}{2}$ for all $N \geq N''$.
	
	To complete the proof, we will show that one can choose $N' \geq N''$ such that $\forall N \geq N'$, $\max_\omega |p_\omega - \hat{p}^N_\omega| > \frac{\epsilon}{2} \Rightarrow I_\phi(p,\hat{p}) > \frac{\rho_0}{N}$.
	First, bound the divergence by
	\begin{align}
		I_{\phi}(p,\hat{p}^N) & = \sum_{\omega=1}^n \hat{p}^N_\omega \phi\left( \frac{p_\omega}{\hat{p}^N_\omega} \right) \nonumber \\
		& = \bar{s} \mathbb{I}_{\bar{s} < \infty} \sum_{\omega \in Z} p_\omega + \sum_{\omega \notin Z} \hat{p}^N_\omega \phi\left( \frac{p_\omega}{\hat{p}^N_\omega} \right) \nonumber \\
		& \geq \bar{s} \mathbb{I}_{\bar{s} < \infty} \sum_{\omega \in Z} p_\omega + \min_{\omega \notin Z} \{\hat{p}^N_\omega\} \cdot \max_{\omega \notin Z} \left\{ \phi \left( \frac{p_\omega}{\hat{p}^N_\omega} \right) \right\} \nonumber \\
		& \geq \bar{s} \mathbb{I}_{\bar{s} < \infty} \sum_{\omega \in Z} p_\omega  + \min_{\omega \notin Z} \{\hat{p}^N_\omega\} \cdot \min\left\{ \phi\left(1+\frac{\epsilon}{2}\right), \phi\left(1-\frac{\epsilon}{2}\right) \right\} \label{eq:asymptotic_proof_phi_substitution} \\
		& \geq \bar{s} \mathbb{I}_{\bar{s} < \infty} \sum_{\omega \in Z} p_\omega + \min_{\omega \notin Z} \left\{ \ptrue_\omega - \frac{\epsilon}{2} \right\} \cdot \min\left\{ \phi\left(1+\frac{\epsilon}{2}\right), \phi\left(1-\frac{\epsilon}{2}\right) \right\} \nonumber,
	\end{align}
	where $\bar{s}\mathbb{I}_{\bar{s} < \infty}$ is the indicator function taking value $\bar{s}$ if $\bar{s} < \infty$ (i.e., if $\phi$ can pop scenarios), and zero otherwise.
	
	Inequality (\ref{eq:asymptotic_proof_phi_substitution}) is true because $\phi \left( \frac{p_\omega}{\hat{p}^N_\omega} \right) \geq \min\left\{ \phi\left( \frac{\hat{p}^N_\omega+\tfrac{\epsilon}{2}}{\hat{p}^N_\omega} \right), \phi\left( \frac{\hat{p}^N_\omega-\tfrac{\epsilon}{2}}{\hat{p}^N_\omega} \right) \right\}$ for at least one $\omega$, and applying the inequalities $\frac{a+\eta}{a} \geq 1 + \eta$ and $\frac{a-\eta}{a} \leq 1-\eta$.
	
	Finally, choose $N'$ to satisfy $\bar{s} \mathbb{I}_{\bar{s} < \infty} \sum_{\omega \in Z} p_\omega + \min_{\omega \notin Z} \left\{ \ptrue_\omega - \frac{\epsilon}{2} \right\} \cdot \min\left\{ \phi\left(1+\frac{\epsilon}{2}\right), \phi\left(1-\frac{\epsilon}{2}\right) \right\} \geq \frac{\rho_0}{N'}$.
\end{proof}

Proposition \ref{prop:weak_conv} shows that the worst-case distributions of (\ref{eq:plp_primal}) converge weakly to $\ptrue$.
In the next theorem, we establish the proof that the optimal value and solution of \plp\ converges to that of the SLP-2 with distribution $\ptrue$ by establishing the epiconvergence of \plp\ to SLP-2.
\begin{theorem}
	\plp\ (\ref{eq:plp_two_stage}) epiconverges to SLP-2 (\ref{eq:slp_first_stage}) with distribution $p = \ptrue$.
\end{theorem}

\begin{proof}
	To establish the epiconvergence, we need only to apply the result of Proposition \ref{prop:weak_conv} to Theorem 3.7 of \cite{dupacova1988asymptotic}, which establishes the epiconvergence of (\ref{eq:plp_primal}) under the evident conditions that the objective function (under the worst-case distribution) is continuous with respect to $\omega$ (because $\omega$ is discrete) and lower semicontinuous and locally lower Lipschitz with respect to $x$ (because \plp is convex).
\end{proof}

%%%%%%%%%%%%%%%%%%%%%%%%%%%%%%%%%%%%%%%%%%%%%%%%%%%%%%%%%%%%%%%%%%%%%%%%%%%%%%%%
\section{Classification of $\phi$-Divergences}
\label{sec:classification}

We present a classification of $\phi$-divergences into four types, resulting from an examination of the limiting behavior of $\phi(t)$ as $t \rightarrow 0$ and $t \rightarrow \infty$.
Different classifications may be suitable to different problem types and desired qualities in the ambiguous model.

As motivation, consider a self-adjoint $\phi$-divergence, which satisfies the relation
\begin{equation} \label{eq:self_adjoint_classification}
	\frac{\phi(t)}{t} = \phi\left(\frac{1}{t}\right),
\end{equation}
and consider $t \rightarrow \infty$.
If both sides of (\ref{eq:self_adjoint_classification}) are finite in the limit, then we see a correspondence between the boundedness of $\phi(t)$ for $t < 1$ and linear growth of $\phi(t)$ for $t > 1$.
On the other hand, infinite limits of (\ref{eq:self_adjoint_classification}) indicate a correspondence between superlinear growth of $\phi(t)$ for $t > 1$ and unboundedness of $\phi(t)$ for $t < 1$.

In the \plp, $\phi$ has arguments given by ratios of probabilities, $\tfrac{p_\omega}{q_\omega}$, and the limits $t \rightarrow 0$ and $t \rightarrow \infty$ correspond to the cases when $p_\omega = 0$ and $q_\omega = 0$, respectively.
Consider each of these limiting cases:
\begin{itemize}
	\item $p_\omega = 0$ but $q_\omega > 0$ ($\lim_{t \searrow 0} \phi(t)$):
	\begin{itemize}
		\item If $\lim_{t \searrow 0} \phi(t) = \infty$, the ambiguity region will never contain distributions with $p_\omega = 0$ but $q_\omega > 0$.
		\item On the other hand, if $\lim_{t \searrow 0} \phi(t) < \infty$, the ambiguity region could contain such a distribution, provided $q_\omega$ is sufficiently small or $\rho$ is sufficiently large.
			We say that such a distribution can \emph{suppress} scenario $\omega$.
	\end{itemize}
	\item $p_\omega > 0$ but $q_\omega = 0$ ($\lim_{t \nearrow 0} \frac{\phi(t)}{t}$):
	\begin{itemize}
		\item If $\lim_{t \nearrow 0} \frac{\phi(t)}{t} = \infty$, the ambiguity region can never contain distributions with $p_\omega > 0$ but $q_\omega = 0$.
		\item On the other hand, if $\lim_{t \nearrow 0} \frac{\phi(t)}{t} < \infty$, the ambiguity region will admit sufficiently small $p_\omega$.
			We say that such a distribution can \emph{pop} scenario $\omega$.
	\end{itemize}
	\item $p_\omega = 0$ but $q_\omega = 0$: Such a situation has no contribution to the divergence, since $0 \phi\left(\tfrac{0}{0}\right) = 0$.
\end{itemize}

These two categories describing suppressing and popping behavior in $\phi$-divergences create four distinct categories.
Examples of divergences in each category are given in Table \ref{tb:phi_categories}.

\begin{table}
	\TABLE
	{
		Examples of $\phi$-divergences fitting into each category.
		\label{tb:phi_categories}
	}
	{\begin{tabular}{l|p{.33\textwidth}p{.33\textwidth}}
		 & Can Suppress Scenarios & Cannot Suppress Scenarios \\
		 \hline
		 Can Pop Scenarios %
			& \parbox{.33\textwidth}{Hellinger,\\Variation Distance} %
			& \parbox{.33\textwidth}{Burg Entropy,\\$\chi^2$-distance} \smallskip \\
		 Cannot Pop Scenarios %
			& \parbox{.33\textwidth}{Kullback-Leibler divergence,\\Modified $\chi^2$-distance} %
			& \parbox{.33\textwidth}{J-Divergence}
	\end{tabular}}
	{}
\end{table}

\subsection{Modeling Considerations When Choosing a Divergence}

We offer the following suggestions for choosing an appropriate $\phi$-divergence classification for the data available.
First, consider whether to choose a distribution that can suppress scenarios.
If the problem scenarios come from high-quality observed data, you may wish to avoid divergences that can suppress scenarios.
However, if the data in poorly sampled, or comes from opinion rather than observation or simulation, the option of suppressing scenarios may result in a solution with better robustness properties.

Next, consider whether to choose a distribution that allows for popping scenarios.
If the problem scenarios come strictly from observation, with little theoretical understanding of the problem, we suggest choosing a divergence that cannot pop scenarios.
However, if the problem scenarios come from a mix of observed/simulated data and expert opinion about scenarios of interest, then divergences that can pop present an interesting modeling choice.
This allows for including interesting but unobserved scenarios, and allowing the mathematical program to assign an appropriate probability to them.

\subsection{Additional Details about Divergences that can Suppress}
\label{ssec:suppress}

Suppressing a scenario is made possible by having $\lim_{t \searrow 0} \phi(t) < \infty$, but there are two further possibilities for such divergences: (1) $\phi'(0) < \infty$ and (2) $\lim_{t \searrow 0} \phi'(t) = \infty$.
This subclassification leads to two different behaviors in $\phi^*(s)$: either (1) $\phi^*(s) = c$ for some constant $c \leq 0$ and  $s <\ \underline{s}$ for $\underline{s} \leq 0$; or (2) $\phi^*(s) \searrow c$ as $s \rightarrow -\infty$ asymptotically, but never reaches the bound.
By the primal-dual variable relation $p_\omega = q_\omega \phi^{*\prime}(s_\omega) = q_\omega \phi^{*\prime}\left(\frac{h_\omega(x) - \mu}{\lambda}\right)$, we see two different possibilities for suppressing scenarios:
\begin{enumerate}
	\item If $\phi'(0) < \infty$, then all scenarios that satisfy the relation $h_\omega(x) < \mu + \underline{s}\lambda$ are suppressed.
		As $\rho$ increases, scenarios tend to be suppressed one at a time.
	\item Otherwise, scenarios can only be suppressed if $s_\omega = -\infty$, which can only occur if $\lambda = 0$.
		Consequently, this solution must also have $\mu = \max_\omega h_\omega(x)$ in order to maintain a probability distribution.
		This also means that all but the most expensive scenario(s) will vanish simultaneously.
		Divergences of this type can be difficult to deal with numerically when suppression occurs.
\end{enumerate}

\subsection{Additional Details about Divergences that can Pop}
\label{ssec:pop}

Divergence that can pop a scenario have $\phi(t)$ that grow linearly as $t \rightarrow \infty$, which causes the existence of an upper bound $\bar{s} = \lim_{t \rightarrow \infty} \frac{\phi(t)}{t}$ on the domain of $\phi^*(s)$.
The primal-dual variable relation specifies $\frac{p_\omega}{q_\omega} = \partial \phi^*(s_\omega)$, but the left-hand side is undefined when $q_\omega = 0$.
Intuitively, we can think of $\frac{p_\omega}{0} = \infty$ if $p_\omega > 0$, and thus popping a scenario can only occur when the right-hand side subderivative also includes $\infty$.
This, in tern, occurs only when $s_\omega = \bar{s}$.

The next proposition makes this statement rigorous.

\begin{proposition} \label{prop:pop}
	A scenario $\omega$ for which $q_\omega = 0$ can only be popped if $s_\omega = \frac{h_\omega(x) - \mu}{\lambda} = \bar{s}$.
\end{proposition}

\begin{proof}
	We present here an abridged derivation of the dual problem (\ref{eq:plp_two_stage}), which can be found in full in \cite{bental2011robust}.
	For this proof, we assume for simplicity that the first stage cost vector $\c = 0$.
	
	We begin with the Legangrian of of (\ref{eq:plp_primal}), $\mathcal{L}(p,\mu,\lambda) = \max_{p \geq 0} \sum_{\omega=1}^n p_\omega h_\omega(x) + (1-\sum_{\omega=1}^n)\mu + (\rho - \sum_{\omega=1}^n q_\omega \phi\left(\frac{p_\omega}{q_\omega}\right)$, for which we generate the dual problem as
	\begin{align}
		\min_{\lambda \geq 0, \mu} \mathcal{L}(p,\mu,\lambda) & = \min_{\lambda \geq 0, \mu} \max_{p \geq 0} \sum_{\omega=1}^n p_\omega h_\omega(x) + (1-\sum_{\omega=1}^n)\mu + (\rho - \sum_{\omega=1}^n q_\omega \phi\left(\frac{p_\omega}{q_\omega}\right) \nonumber \\
		& = \min_{\lambda \geq 0, \mu} \mu + \rho\lambda + \sum_{\omega=1}^n \max_{p_\omega \geq 0} p_\omega (h_\omega(x) - \mu) - q_\omega \lambda \phi\left(\frac{p_\omega}{q_\omega}\right) \label{eq:pop_proof_detail_1} \\
		& = \min_{\lambda \geq 0, \mu} \mu + \rho\lambda + \lambda \sum_{\omega=1}^n q_\omega \max_{t \geq 0} \left\{ \frac{h_\omega(x) - \mu}{\lambda} t - \phi(t) \right\} \label{eq:pop_proof_detail_2} \\
		& = \min_{\lambda \geq 0, \mu} \mu + \rho\lambda + \lambda \sum_{\omega=1}^n q_\omega \phi^*\left(\frac{h_\omega(x) - \mu}{\lambda}\right), \nonumber
	\end{align}
	where $t = \frac{p_\omega}{q_\omega}$.
	
	To account for the possibility that $q_\omega = 0$ and demonstrate popping behavior, equality (\ref{eq:pop_proof_detail_2}) must be modified slightly.
	Consider a term in the summation in (\ref{eq:pop_proof_detail_1}) for which $q_\omega = 0$,
	\begin{align}
		\max_{p_\omega \geq 0} p_\omega (h_\omega(x) - \mu) - q_\omega \lambda \phi\left(\frac{p_\omega}{q_\omega}\right) & = \max_{p_\omega \geq 0} p_\omega (h_\omega(x) - \mu) - 0 \lambda \phi\left(\frac{p_\omega}{0}\right) \nonumber \\
		& = \max_{p_\omega \geq 0} p_\omega (h_\omega(x) - \mu) - \lambda p_\omega \bar{s} \nonumber \\
		& = \max_{p_\omega \geq 0} p_\omega \left( h_\omega(x) - \mu - \bar{s} \lambda \right). \label{eq:pop_proof_condition}
	\end{align}
	The behavior of (\ref{eq:pop_proof_condition}) depends on the sign of $\left( h_\omega(x) - \mu - \bar{s} \lambda \right)$, or equivalently, relation between $\frac{h_\omega(s) - \mu}{\lambda}$ and $\bar{s}$.
	There are three cases:
	\begin{description}
		\item[Case 1: $\frac{h_\omega(s) - \mu}{\lambda} > \bar{s}$] selects $p_\omega = \infty$, which induces the constraint $\frac{h_\omega(s) - \mu}{\lambda} \leq \bar{s}$ for scenarios with $q_\omega = 0$.
		\item[Case 2: $\frac{h_\omega(s) - \mu}{\lambda} < \bar{s}$] selects $p_\omega = 0$.
		\item[Case 3: $\frac{h_\omega(s) - \mu}{\lambda} = \bar{s}$] places no restrictions on the value of $p_\omega$, since (\ref{eq:pop_proof_condition}) is identically zero. %\qedhere
	\end{description}
\end{proof}

\begin{remark}
	Because $s_\omega = \frac{h_\omega(x) - \mu}{\lambda}$, $s_\omega \leq \bar{s}$ for all $\omega$ and $s_\omega = \bar{s}$ for any popped scenarios, only the most expensive scenario could be popped.
\end{remark}

\begin{remark}
	Finding the probability of the popped scenario cannot be done by differentiating $\phi^*$ as with other scenarios, thus the probability must be calculated with $\sum_\omega p_\omega = 1$.
\end{remark}

%%%%%%%%%%%%%%%%%%%%%%%%%%%%%%%%%%%%%%%%%%%%%%%%%%%%%%%%%%%%%%%%%%%%%%%%%%%%%%%%
\section{Some Special $\phi$-Divergences}
\label{sec:special_phi}

The class of $\phi$-divergence constrained problems includes some interesting special cases, which we document here.

\begin{example}{CVaR}
	The coherent risk measure Conditional Value-at-Risk (CVaR) is well studied in financial applications.
	$\text{CVaR}_\beta$ is equivalent to the $\phi$-divergence constrained problem with 
	\[
		\phi(t) = \
		\begin{cases}
			0 & 0 \leq t \leq \frac{1}{1-\beta} \\
			\infty & \text{otherwise},
		\end{cases}
	\]
	for $0 < \beta < 1$.
\end{example}

\begin{remark}
	We see that $\phi(0) = 0$, indicating that CVaR will suppress some scenarios.
	This appears in the definition of CVaR as the positive part in the expected value, $e{[h(x)-\mu]^+}$.
	Scenarios cannot be popped because the expectation is taken with respect to the nominal distribution.
\end{remark}

The CVaR $\phi$-divergence is bounded above, which leads to the question of what happens when a divergence is bounded below.
\begin{example}[``Reverse'' CVaR]
	The $\phi$-divergence constrained problem with
	\[
		\phi(t) = \
		\begin{cases}
			0 & t \geq 1-\beta \\
			\infty & t < 1-\beta,
		\end{cases}
	\]
	for $0 < \beta < 1$ is equivalent to $\beta \sup_\omega h_\omega(x) + (1-\beta)\e{h(x)}$, where the expectation is taken with respect to the nominal distribution $q$.
\end{example}

\begin{remark}
	$\lim_{t \rightarrow \infty} \frac{\phi(t)}{t} = 0$, indicating that this divergence will pop scenarios.
	This behavior appears due to the $\sup_\omega h_\omega(x)$.
	However, $\phi(0) = \infty$ indicates that scenarios will not be suppressed, which is demonstrated by the expectation term $\e{h(x)}$, which takes into account every scenario with positive nominal probability.
\end{remark}

An objective function taking a weighted sum of expected value and CVaR often comes up in practice.
The next theorem shows how to generate a convex combination of expectation and CVaR.

\begin{example}[Conbination CVaR and Expectation]
	The $\phi$-divergence constrained problem with
	\[
		\phi(t) = 
		\begin{cases}
			0 & 1-\alpha \leq t \leq \frac{1}{1-\beta} \\
			\infty & \text{otherwise},
		\end{cases}
	\]
	for $\alpha,\beta \in (0,1)$ is equivalent to 
	\[
		(1-\alpha)\e{h(x)} + \alpha \mbox{CVaR}_{\frac{\beta}{\alpha(1-\beta)+\beta}}[h(x)].
	\]
\end{example}

\begin{remark}
	This divergence will neither pop (because both the expectation and CVaR term are taken with respect to the nominal distribution) nor suppress (because the expectation term includes every scenario).
\end{remark}

%%%%%%%%%%%%%%%%%%%%%%%%%%%%%%%%%%%%%%%%%%%%%%%%%%%%%%%%%%%%%%%%%%%%%%%%%%%%%%%%
\section{Decomposition-Based Solution Method}
\label{sec:soln_algorithm}

As the model gets larger, as in our water application presented in Section~\ref{sec:comp_results}, a direct solution of \plp\ becomes computationally expensive. 
Decomposition-based methods could significantly reduce the solution time and allow for larger problems to be solved efficiently. In this section, we propose a Bender's decomposition-based method for solving \plp.
The algorithm removes feasibility constraint (\ref{eq:plp_feas_constraint})  and exchanges it with a series of feasibility cuts in the first-stage problem.
The master problem is given by
\begin{align}
	\min_{\x,\lambda,\mu} \ & \c\x + \mu + \rho \lambda + \theta \label{eq:master_problem}\\
	\st \ & A\x = \b \nonumber \\
	& \theta \geq T_j (\x,\mu,\lambda)^T + t_j, \ j \in J \nonumber \\
	& \mu + \bar{s}\lambda \geq M_k \x + m_k, \ k \in K \nonumber \\
	& \x,\lambda \geq 0, \nonumber
\end{align}
where $T_j (\x,\mu,\lambda)^T + t_j$ are the objective cuts, $M_k \x + m_k$ are the feasibility cuts on constraint (\ref{eq:plp_feas_constraint}) if $\bar{s} = \lim_{t \rightarrow \infty} \frac{\phi(t)}{t} < \infty$, and $J$ and $K$ are the sets of objective and feasibility cuts, respectively.
The proposed algorithm is shown below.

% \begin{figure}
% 	\FIGURE
% 	{%
	\begin{algorithmic}
		\State Initialize $z_l = -\infty, z_u = \infty$
		\State Solve first stage (\ref{eq:master_problem}) with $\theta = 0$  to generate $\x$
		\State Solve all second stage scenarios $h_\omega(\x)$ (\ref{eq:slp_second_stage})
		\State Initialize $\lambda \gets 1$, $\mu$ so that $\frac{h_\omega(\x) - \mu}{\lambda} < \bar{s}$
		\State Generate initial objective cut
		\While{$z_u - z_l \geq \texttt{TOL}\min\{|z_u|,|z_l|\}$}
			\State Solve master problem (\ref{eq:master_problem}), get $\x$,$\lambda$,$\mu$,$\theta_M$
			\State Solve sub-problems $h_\omega(\x)$ (\ref{eq:slp_second_stage})
			\State $\theta_{\text{true}} \gets \sum_{\omega=1}^n q_\omega h_\omega(\x,\lambda,\mu)$
			\If{$\frac{h_\omega(\x) - \mu}{\lambda} > \bar{s}$}
				\State Generate feasibility cut
				\State Find $\mu$ so that $\frac{h_\omega(\x) - \mu}{\lambda} < \bar{s}$
			\Else
				\State $z_l \gets$ master optimal cost $\c\x + \mu + \bar{N}\lambda + \theta_{\text{true}}$
			\EndIf
			\State Generate objective cut
			\If{$\c\x + \mu + \bar{N}\lambda + \theta_{\text{true}} < z_u$}
				\State $z_u \gets \c\x + \mu + \bar{N}\lambda + \theta_{\text{true}}$
				\State $\x_\text{best} \gets \x, \lambda_\text{best} \gets \lambda, \mu_\text{best} \gets \mu$
				\State $p_\omega \gets \phi^{*\prime}(\tfrac{h_\omega(\x) - \mu}{\lambda})$ for $\omega = 1, \dots, n$
			\EndIf
		\EndWhile
	\end{algorithmic}
% 	}
% 	{
% 		Proposed Bender's Decomposition algorithm for solving \plp
% 		\label{fig:algorithm}	
% 	}
% 	{}
% \end{figure}

\subsection{Objective Cuts}

Let $(\xh,\mh, \lh)$ be the candidate solution from the master problem (\ref{eq:master_problem}), and let 
\[
	h^\dagger_\omega(\x,\mu,\lambda) = \lambda \phi^*\left(\frac{h_\omega(\x) - \mu}{\lambda}\right)
\]
be the nonlinear portion of the objective function which will be used to generate the objective cuts.
An objective cut can be computed by solving the SLP-2 subproblems $h_\omega(\xh)$ along with optimal dual solutions $\pi^{*,\omega}$ to each second-stage problem, and using these to compute the partial (sub)derivatives of the \plp\ subproblems as
\begin{align*}
	\dfrac{\partial h^\dagger_\omega}{\partial \x}(\xh,\mh,\lh) & = \phi^{*\prime}(s_\omega(\xh,\mh,\lh)) \pi^{*,\omega}B^\omega \\
	\dfrac{\partial h^\dagger_\omega}{\partial \mu}(\xh,\mh,\lh) & = -\phi^{*\prime}(s_\omega(\xh,\mh,\lh)) \\
	\dfrac{\partial h^\dagger_\omega}{\partial \lambda}(\xh,\mh,\lh) & = \phi^*(s_\omega(\xh,\mh,\lh)) - \phi^{*\prime}(s_\omega(\xh,\mh,\lh))s_\omega(\xh,\mh,\lh),
\end{align*}
where $s_\omega(\xh,\mh,\lh) = \frac{h_\omega(\xh) - \mh}{\lh}$.
Recall that $h_\omega(\x) = \min_{y^\omega \geq 0} \{\k^\omega y^\omega | D^\omega y^\omega = \d^\omega + B^\omega \x\}$.
The cuts are then given by
\begin{align*}
	T_j^\omega & = 
	\left( \begin{array}{ccc}
		\phi^{*\prime}(s_\omega \pi^{*,\omega}B^\omega, 
			 & -\phi^{*\prime}(s_\omega, 
			 & \phi^*(s_\omega - \phi^{*\prime}(s_\omega s_\omega
	\end{array} \right) \\
	t_j^\omega & = \lh \phi^{*\prime}(s_\omega)\left[s_\omega - \frac{\pi^{*,\omega}B^\omega\xh - \mh}{\lh}\right].
\end{align*}

For the single-cut master problem proposed, $T_j = \sum_\omega q_\omega T_j^\omega$ and $t_j = \sum_\omega q_\omega t_j^\omega$.

\subsection{Feasibility Cuts}
After the subproblems $h_\omega(\xh)$ are solved, it may be the case that $\frac{h_\omega(\xh)-\mh}{\lh} < \bar{s}$ for some $\omega$, rendering $\mh$ and $\lh$ infeasible.
This is corrected using the feasibility problem
\begin{align*}
	U_\omega(\x,\mu,\lambda) = \min_{y^\omega \geq 0, z \geq 0} \ & z \\
	\st \ & z + \bar{s}\lambda + \mu - q^\omega y^\omega \geq 0 \\
	& D^\omega y^\omega = d^\omega + B^\omega x,
\end{align*}
which is solved by $z = h_\omega(x) - \bar{s}\lambda - \mu$.
The subdifferentials can be easily found as $\frac{\partial z^*}{\partial \x} = \pi^{*,\omega} B^\omega$, $\frac{\partial z^*}{\partial \mu} = -1$, and $\frac{\partial z^*}{\partial \lambda} = -\bar{s}$.
Then for infeasible candidate solution $(\xh,\lh,\mh)$ we get the inequality
\begin{align*}
	U_\omega(\x,\mu,\lambda) \geq \pi^{*,\omega}B^\omega(\x-\xh) - (\mu -\mh) - \bar{s}(\lambda - \lh) + (h_\omega(\xh) - \mh - \bar{s}\lh),
\end{align*}
and setting $U_\omega(\x,\mu,\lambda) = 0$ to find a feasible solution gives the feasibility cut
\[
	\mu + \bar{s} \lambda \geq \pi^{*,\omega}B^\omega \x + (h_\omega(\xh) - \pi^{*,\omega}B^\omega\xh).
\]

\subsection{Computational Enhancements}

In order to enhance the performance of the above decomposition-based algorithm, we made some adjustments.
First, we included an $L_\infty$-norm trust region which is scaled up (by a factor of $3$) or down (by a factor of $\tfrac{1}{4}$) when the trust region inhibits finding the optimal solution or when the polyhedral lower approximation is far from the second-stage expected cost, respectively.
The trust region is an implementation of Algorithm 4.1 in \cite{nocedal1999numerical}.

Because we are also interested in the worst-case probabilities given in the primal variables and not computed directly, we include a second tolerance as a stopping condition, ensuring that $\left| 1 - \sum_{i=1}^n p_\omega \right| < \texttt{TOL2}$ when the algorithm is completed.
This must be satisfied in addition to the original condition $z_u - z_l < \texttt{TOL}\min\{|z_u|,|z_l|\}$.

%%%%%%%%%%%%%%%%%%%%%%%%%%%%%%%%%%%%%%%%%%%%%%%%%%%%%%%%%%%%%%%%%%%%%%%%%%%%%%%%
\section{Application and Computational Results} \label{sec:comp_results}

\subsection{Computational Behavior of $\phi$-Divergences}
\label{ssec:computational_classification}

Section \ref{sec:classification} discussed a method of classifying $\phi$-divergences into four types based on two criteria: whether a $\phi$-divergences allows distributions that suppress certain scenarios (i.e., allows $p_\omega = 0$ while $q_\omega > 0$); and whether a $\phi$-divergence allows distributions that pop a scenario that is impossible in the nominal distribution (i.e., allows $p_\omega > 0$ while $q_\omega = 0$).
In this section we present computational results demonstrating the suppressing and popping behavior of different divergences.
The categorization of several $\phi$-divergences can be found in Table \ref{tb:phi_categories}.

All plots in this section were computed using a toy problem with $n = 4$ scenarios to allow for a clear picture of how the probabilities change with $\rho$.
In this problem, the second scenario (displayed in green in Figures \ref{fig:suppress} and \ref{fig:pop}) is the most costly, and thus the only candidate for popping if its nominal probability vanishes.
Furthermore, the second scenario will eventually have unit probability in suppressing divergences.
All scenarios are assumed to have a single observation, unless popping behavior is being demonstrated.

We begin with an examination of the divergences that can suppress scenarios in Figure \ref{fig:suppress}.
Figure \ref{fig:suppress} shows how the worst-case distribution changes with $\rho$ for both the Modified $\chi^2$ divergence (left) and the Kullback-Leibler divergence (right).
As shown in Section \ref{ssec:suppress}, the Modified $\chi^2$ distance suppresses scenarios one-at-a-time, starting with the least expensive; while the Kullback-Leibler divergence will suppress all but the most costly scenario simultaneously.

\begin{figure}
	\FIGURE
	{%
		\includegraphics*[width=.5\textwidth]{images/mchi2}%
		\includegraphics*[width=.5\textwidth]{images/kl}%
	}
	{
		Examples of distributions that can suppress: Modified $\chi^2$ distance (left) and Kullback-Leibler Divergence (right).
		Notice that the Modified $\chi^2$ distance suppresses scenarios one at a time, while the Kullback-Leibler Divergence suppresses all three lower-cost scenarios simultaneously.
		\label{fig:suppress}
	}
	{}
\end{figure}

An example of a $\phi$-divergence that can pop, the Burg entropy, is given in Figure \ref{fig:pop}.
The left plot in Figure \ref{fig:pop} demonstrates the worst-case distribution assuming that all scenarios have a single observation.
The right plot shows the worst-case distribution when all scenarios but the most costly have a single observations, while the most costly scenario is unobserved.
Notice, in particular, that the probability of the most costly scenario becomes small as $\rho$ decreases.
Other divergences that can pop but not suppress look qualitatively similar.

\begin{figure}
	\FIGURE
	{%
		\includegraphics*[width=.5\textwidth]{images/burg}%
		\includegraphics*[width=.5\textwidth]{images/burg_zero}%
	}
	{
		Example of a distribution that can pop: the Burg entropy.
		The left figure shows the worst-case distribution assuming that all scenarios have a single observation.
		The right figure demonstrates the most costly scenario entering the worst-case distribution despite it having no observations.
		\label{fig:pop}
	}
	{}
\end{figure}

\subsection{Generalized Network Water Model} 
\label{ssec:network_model}

We applied \plp\ to a multi-period generalized network flow model of Colorado River water allocation in Tucson, defined by a set of nodes and directed arcs $(N,\: A)$.
The nodes represent available water supply from the Colorado River, water treatment plants, reservoirs, and water demand sites.
The arcs represent the conveyance system (pipes, etc.) that carry water between the nodes. 
Water can be stored in between time periods in reservoirs to meet future demands. 
The model aims to find the minimal cost water flows considering energy, treatment, storage, and transportation costs over the planning period. 
Generalized network water allocation models have been used to find water allocations and delivery reliabilities and to assess values of different water use operations; see, e.g., \cite{draper_etal_03}. 

Water flows on arc $(i,j) \in A$ during time period $t = 1, \dots, P$ are represented by decisions $x_{ijt}$.
Each arc $(i,j) \in A$ and time period $t$ has a unit cost $c_{ijt}^x$, loss coefficient $0 \leq a_{ijt} \leq 1$ to account for evaporation, leakage from the pipes, etc., and bounds on the flow $l_{ijt}^x \leq x_{ijt} \leq u_{ijt}^x$.
Each node $j \in N$ has a supply/demand for time period $t$, $b_{jt}$.
Nodes representing reservoirs are able to store water between time periods.
Stored water available at node $j$ at the beginning of time period $t$ is $s_{jt}$, with associated cost $c_{jt}^s$ and bounds $l_{jt}^s \leq s_{jt} \leq u_{jt}^s$.
Finally, water released into the environment from node $j$ in period $t$ is given by $r_{jt}$, with bounds $l_{jt}^r \leq r_{jt} \leq u_{jt}^r$.
The deterministic model is a multi-period generalized network flow model of the form
\begin{align*}
	\min_{x,s,r} \ & \sum_{(i,j) \in A} \sum_{t=1}^P c_{ijt}^x x_{ijt} + \sum_{j \in N} \sum_{t=1}^P c_{jt}^s s_{jt}\\
	\st \ & \sum_{i \in N} x_{jit} + s_{j,t+1} + r_{jt} = \sum_{i \in N} a_{ijt} x_{ijt} + s_{jt} + b_{jt}, \ \ \forall j,t \\
	& l_{ijt}^x \leq x_{ijt} \leq u_{ijt}^x,\ \ \ \forall i,j,t \\
	& l_{jt}^s \leq s_{jt} \leq u_{sjt}^s, \ \ \ \forall j,t \\
	& l_{jt}^r \leq r_{jt} \leq u_{sjt}^r, \ \ \ \forall j,t.
\end{align*}

The model is converted to a two-stage stochastic model with $P_1$ periods in the first stage and $P-P_1$ stages in the second stage.
We assume that the supplies and demands are uncertain, as well as the bounds on the decision variables.
\begin{align}
	\min_{(x,s,r) \in L^1} \ & \sum_{(i,j) \in A} \sum_{t=1}^{P_1} c_{ijt}^x x_{ijt} + \sum_{j \in N} \sum_{t=1}^{P_1} c_{jt}^s s_{jt} + \sum_{\omega=1}^n p_\omega h_\omega(s) \label{eq:gen_network_two_stage} \\
	\st \ & \sum_{i \in N} x_{jit} + s_{j,t+1} + r_{jt} = \sum_{i \in N} a_{ijt} x_{ijt} + s_{jt} + b_{jt},\ \ \ \ \forall j, 1 \leq t \leq P_1, \notag
\end{align}
where the second stage problems
\begin{align}
	h_\omega(s) = \min_{(x,s,r) \in L^2_\omega} \ & \sum_{(i,j) \in A} \sum_{t=P_1+1}^{P} c_{ijt}^x x_{ijt} + \sum_{j \in N} \sum_{t=P_1+1}^{P} c_{jt}^s s_{jt} \label{eq:gen_network_second_stage} \\
	\st \ & \sum_{i \in N} x_{jit} + s_{j,t+1} + r_{jt} = \sum_{i \in N} a_{ijt} x_{ijt} + s_{jt} + b_{jt}^\omega, \ \ \ \ \forall j, P_1+1 \leq t \leq P, \notag
\end{align}
and $L^1$ and $L^2_\omega$ represent the feasible regions defined by the lower and upper variable bounds.

The model has a total of $P = 41$ time periods, representing years 2010--2050.
For each time period, the network has 62 nodes representing demand for potable and nonpotable (reclaimed) water, pumps, water treatment plants, and the available water supply from the Colorado River.
The network in each time period has 102 arcs, representing the pipe network carrying the water between the nodes physically and connecting the network to the five reservoirs that connect the time stages in the model.
We use $P_1 = 10$ time periods for the first stage.
Uncertainty in the second stage takes the form of uncertain population (thus, demand for water) and supply of water.
There are a total of 4 scenarios considered in this test instance: (i) high population, high supply, (ii) high population, low supply, (iii) low population, high supply, and (iv) low population, low supply.
Each scenario is assumed to have five observations.
The high population scenarios are more costly as the system needs to meet demand or pay for unmet demand.
The low population scenarios, on the other hand, are not as costly. 
% The supply variability seems to have a nominal effect.
We applied the decomposition-based solution algorithm presented in Section \ref{sec:soln_algorithm} to solve this model.% and selected tolerances $\texttt{TOL} = 10^{-5}$ and $\texttt{TOL2} = 10^{-3}$ for our computational experiments.

\subsection{Computational Results}

Figure \ref{fig:worst_case} shows how the worst-case distribution changes with $\gamma'$.
When $\gamma'$ is close to $1$, we use the maximum likelihood distribution, which has equal $\tfrac{1}{4}$ probabilities on each of the four scenarios.
As $\gamma'$ is decreased, the ambiguity set increases, and the worst-case distribution used by \plp\ changes.
It gives higher than $\tfrac{1}{4}$ probability to the two high-population scenarios and lower than $\tfrac{1}{4}$ probability to the two low-population scenarios, making the solution more robust to costly scenarios.
Note that the scenarios fall into two similar pairs because the cost of each scenario depends strongly on the projected demand but only weakly on the projected supply of Colorado River water.
A closer look at the optimal solutions reveals that as $\gamma'$ is decreased, or as robustness is increased, the solution uses more and more reclaimed water (treated wastewater that is reused for nonpotable purposes such as irrigation) in an effort to meet demands in a least-costly way.

\begin{figure}
	\FIGURE
	{%
		\includegraphics*[width=.5\textwidth]{images/worst_case_probability}%
		\includegraphics*[width=.5\textwidth]{images/water_prob_decrease}%
	}
	{
		(a) Worst-case distribution for the likelihood robust water allocation problem.
		(b) Probability that an additional sample causes a decrease in worst-case expected cost for the likelihood robust water allocation problem.
		The red line shows the upper bound probability $\tfrac{N_L}{N}$.
		\label{fig:worst_case}
		\label{fig:water_prob_decrease}
	}
	{}
\end{figure}

%\subsection{Value of Data}

The results of the water model were then analyzed with the value of data techniques from Section \ref{ssec:value}.
Figure \ref{fig:water_prob_decrease} shows the estimated probability that an additional sample will remove the worst-case distribution from the likelihood region, resulting in a lower-cost solution. %, analogous the results of Figure \ref{fig:prob_cost_decrease_gp}.
The dashed line in Figure \ref{fig:water_prob_decrease} depicts the computed values of $\frac{N_L}{N}$, which provide an upper bound on the estimated probabilities. 
Because the low-population scenarios have lower costs, an additional sample of either low-population scenario will result in a lower expected cost.
This is what we see through most of the computed values of $\gamma'$, with $\frac{N_L}{N} = 0.5$, indicating that the sufficient condition (\ref{eq:cost_decrease_cond}) was satisfied for both low-population scenarios.
For extremely large values of $\gamma'$---above $0.97$---we see the ratio $\frac{N_L}{N}$ quickly drops to zero.
This occurs because (\ref{eq:cost_decrease_cond}) only compares scenario probabilities in the empirical ($N_\omega / N$) and worst-case ($p_\omega$) distributions and does not use the computed cost of the scenarios.
As $\gamma'$ increases and the ambiguity set shrinks, the worst-case probabilities become so close the empirical probabilities that (\ref{eq:cost_decrease_cond}) can no longer be satisfied, resulting in $\frac{N_L}{N}$ decreasing to zero.

%%%%%%%%%%%%%%%%%%%%%%%%%%%%%%%%%%%%%%%%%%%%%%%%%%%%%%%%%%%%%%%%%%%%%%%%%%%%%%%%
\section{Conclusion and Future Work}
\label{sec:concl}

In this paper, we proposed an implementation of the $\phi$-divergence constrained distributionally robust optimization method of \cite{bental2011robust} for two-stage stochastic linear programs with recourse, creating a two-stage $\phi$-divergence linear program with recourse, denoted \plp.
The \plp uses a $\phi$-divergence to define an ambiguity set of probability distributions, possibly using observed data, and optimizes the worst-case expected cost with respect to this ambiguity set.
We provided a classification of $\phi$-divergences that may be useful in determining which is most appropriate in practice for several model types.
We provided a computationally simple method to estimate the probability that an additional sample will produce a likelihood ambiguity set that does not contain the current worst-case distribution and will result in a lower-cost solution. 
We have also provided a Bender's decomposition-based solution algorithm for the \plp\ and applied this method to planning future water distribution in Tucson, Arizona.

Our future work includes the following. We plan to augment the existing model first with a richer set of second-stage scenarios.
In addition to more varied estimates for future population, we will integrate climate change predictions into the model to generate scenarios for future water supply from the Colorado River. 
This model is intended to include a facility location problem to determine the best places for an additional waste water treatment plants to increase the use of reclaimed water in the most cost-efficient manner.

% Acknowledgments here
\ACKNOWLEDGMENT{%
This work has been partially supported by the National Science Foundation through grant CMMI-1345626. We also gratefully acknowledge support provided by a Water Sustainability Program Fellowship through the Technology and Research Initiative Fund at the University of Arizona.
}% Leave this (end of acknowledgment)


% Appendix here
% Options are (1) APPENDIX (with or without general title) or 
%             (2) APPENDICES (if it has more than one unrelated sections)
% Outcomment the appropriate case if necessary
%
% \begin{APPENDIX}{<Title of the Appendix>}
% \end{APPENDIX}
%
%   or 
%
% \begin{APPENDICES}
% \section{<Title of Section A>}
% \section{<Title of Section B>}
% etc
% \end{APPENDICES}


% References here (outcomment the appropriate case) 

% CASE 1: BiBTeX used to constantly update the references 
%   (while the paper is being written).
%\bibliographystyle{ijocv081} % outcomment this and next line in Case 1
\bibliography{love_lro} % if more than one, comma separated

% CASE 2: BiBTeX used to generate mypaper.bbl (to be further fine tuned)
%\input{mypaper.bbl} % outcomment this line in Case 2

%\section*{Author Biographies}

%\noindent {\bf DAVID LOVE} is a graduate student in the Graduate Interdisciplinary Program in Applied Mathematics at the University of Arizona.
%His research interests include distributionally robust stochastic programming and water resources management.
%His email address is \url{dlove@math.arizona.edu} and his web page is \url{http://math.arizona.edu/~dlove}.

%\bigskip

%\noindent {\bf G\"{U}ZIN BAYRAKSAN} is an Associate Professor of Integrated Systems Engineering at the Ohio State University.
%Her research interests include Monte Carlo sampling methods for stochastic programming and applications to water resources.
%Her email address is \url{bayraksan.1@osu.edu} and her web page is \url{http://www-iwse.eng.ohio-state.edu/biosketch_GBayraksan.cfm}

\end{document}

